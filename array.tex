\chapter{数组}\label{fortran_array}

迄今为止我们折腾的东东都是标量(scalar),那都是小case,大case是数组(array).用非常严谨的方式来讨论数组,以我这语文水平肯定是不行滴,讲着讲着同志们就头疼,所以我接下来讲的内容会不太严谨,但对于实际应用来说肯定是不成问题的.

数组是形如$a\!:\!\{j_1,\dots,k_1\}\!\times\!\dots\!\times\!\{j_n,\dots,k_n\}\rightarrow\mathbb{C},(s_1,\dots,s_n)\mapsto a_{s_1\dots s_n}$的东东.正整数$n$称为数组$a$的维数/秩(rank).任意$i\in\{1,\dots,n\}$,$j_i$和$k_i$称为数组$a$的第$i$个维度(dimension)的下界(lower-bound)和上界(upper-bound),$k_i-j_i+1$称为数组$a$的第$i$个维度的长度(extent).矢量$(k_1-j_1+1,\dots,k_n-j_n+1)$称为数组$a$的形状(shape),其本身是一维数组.$\prod_{i=1}^n(k_i-j_i+1)$称为数组$a$的大小(size).标量可以当成维数为$0$的数组.

形如$a_{s_1\dots s_n}$的东东称为数组$a$中的元素(element).任意$i\in\{1,\dots,n\}$,称$s_i$为$a_{s_1\dots s_n}$的第$i$个下标(subscript)或第$i$个索引(index).同一个数组中的所有元素都有相同的类型和种别.

数组中的元素有一个被规定死的排列顺序: 任意$a_{s_1\dots s_n}$和$a_{t_1\dots t_n}$,当$s_{i+1}=t_{i+1},\dots,s_n=t_n$时,若$s_i<t_i$,则$a_{s_1\dots s_n}$在$a_{t_1\dots t_n}$前面,若$s_i>t_i$,则$a_{s_1\dots s_n}$在$a_{t_1\dots t_n}$后面\footnote{这和Matlab的默认情形一样,但和C/Python的默认情形不一样!}.

\section{数组声明}\label{fortran_array_specification}

用数组之前当然要声明它.数组的声明和标量的声明还是很像的,也是需要给一个数据类型(整型/实型/复型),还可以给一个种别.数组的类型和种别就是数组中每一个元素的类型和种别.

为表明声明的变量是个数组,还需给出$j_1,\dots,j_n,k_1,\dots,k_n$.有两种方法可行,一种是在变量名后加\ttt{(j1:k1,...,jn:kn)},另一种是在类型和种别后加\ttt{,dimension(j1:k1,...,jn:kn)}.所有\ttt{ji},\ttt{ki}都必须是整型常量或整型常量表达式\footnote{
    就是只含整型常量的表达式.
}.可以省略任意\ttt{ji:},如果省略\ttt{ji:},则\ttt{ji}为\ttt{1}\footnote{这和Matlab的默认情形一样,但和C/Python的默认情形不一样!}.

请大家猛戳\href{https://fortran-lang.org/learn/quickstart/arrays_strings#array-declaration}{这个链接}获取数组声明示例.

选择哪种声明方式呢?一般大家喜欢用第一种方式,毕竟少打几个字,而且看着比较简洁.但如果是要一口气声明一堆一样的数组\footnote{
    方法同标量声明.
},这时第一种方式反而不好使了,大家就喜欢用第二种方式.

和字符串一样,有些时候我们并不清楚到底要用一个什么样的数组,这时我们就可以和字符串一样,先在类型和种别后加\ttt{,allocatable},然后再在变量名后加\ttt{(:,...,:)}或在类型和种别后加\ttt{,dimension(:,...,:)},其中\ttt{:}的个数等于数组的维数,这样我们就搞出一个延迟形状数组(deferred-shape array).假如我们声明的数组叫\ttt{a},接下来我们就可以像延迟长度字符串那样,用\ttt{allocate(a(j1:k1,...,jn:kn))}把$j_1,\dots,j_n,k_1,\dots,k_n$确定下来(仍可省略\ttt{ji:}),用\ttt{deallocate(a)}将$j_1,\dots,j_n,k_1,\dots,k_n$重新变成未定义的,然后再来个\ttt{allocate(a(j1:k1,...,jn:kn))}\dots

请大家猛戳\href{https://fortran-lang.org/learn/quickstart/arrays_strings#allocatable-dynamic-arrays}{这个链接}获取延迟形状数组声明示例.

\section{数组构造}

数组构造器(array constructor)是一维数组常量,形如\ttt{[e1,...,em]},其中\ttt{ei}可以是标量也可以是数组.若将\ttt{[e1,...,em]}记为$(a_1,\dots,a_n)$,\ttt{ei}的大小记为$S_i$\footnote{
    标量的大小当然是$1$.
},则$a_{(S_1+\dots+S_{i-1}+l)}$为\ttt{ei}的第$l$个元素.如果\ttt{ei}维数大于$1$,则按本章开头``数组元素顺序''找到第$l$个元素.

直接了当的说法是: 把所有\ttt{ei}重新排成一维数组并保证元素顺序不变,然后首尾相接拼起来.

在数组构造器\uline{中}还可以使用一种神奇的隐式do循环(implied \ttt{do} loop),这玩意儿形如\ttt{((...(e(i1,...,in),i1=p1,q1,r1)...),in=pn,qn,rn)},里头\ttt{e(i1,...,in)}是个数据实体,通常是个含有\ttt{i1},\dots,\ttt{in}的表达式,\ttt{i1},\dots,\ttt{in}是一堆事先声明过的整型变量,整个隐式do循环相当于\uline{一维}数组.下面这样的两个程序总是等价的\footnote{
看懂这两个程序,同志们可能需要先读\ref{fortran_array_assignment}节.
}.自然任意\ttt{,ri}都可以省略(默认为\ttt{1}).
\begin{lstlisting}
program normal_do_loop
    ...
    integer :: i1
    ...
    integer :: in
    integer :: i
    i = 0
    do in = pn,qn,rn
        ...
            do i1 = p1,q1,r1
                i = i+1
                a(i) = e(i1,...,in) ! a is [a(1),...,a(S)].
            end do
        ...
    end do
    ! Now i == S.
end program normal_do_loop
\end{lstlisting}
\begin{lstlisting}
program implied_do_loop
    ...
    integer :: i1
    ...
    integer :: in
    a = [((...(e(i1,...,in),i1=p1,q1,r1)...),in=pn,qn,rn)]
end program implied_do_loop
\end{lstlisting}

隐式do循环属于比较高级的语法,还是稍稍让人不好理解滴.幸好不用隐式do循环也总是可以完成任务,所以可以干脆不用.Python中也有一个类似的叫列表推导式的东东,不过当年老师好像根本没讲过,我也只是在耍帅时会用这玩楞.

用Ifx的话,可以用\ttt{p:q:r}代替\ttt{(i1,i1=p,q,r)},当然\ttt{:r}可以省略.这是Ifx的器规,Gfortran是不认的\footnote{
所以造轮子时最好不用这东东.
}.\ttt{(i1,i1=p,q,r)}这样的一重简单隐式do循环比较常用,是要掌握的.我再把上一对示例程序的简单情形重新列一下,其中\ttt{a(i) = i1}的意思就是令$a_i$为那时的$i_1$.
\begin{lstlisting}
program normal_do_loop
    ...
    integer :: i1
    integer :: i
    i = 0
    do i1 = p,q,r
        i = i+1
        a(i) = i1
    end do
end program normal_do_loop
\end{lstlisting}
\begin{lstlisting}
program implied_do_loop
    ...
    integer :: i1
    a = [(i1,i1=p,q,r)]
end program implied_do_loop
\end{lstlisting}
举个更具体的例子: 输出$1$到$9$中奇数的平方.
\begin{lstlisting}
program main
    integer :: i,odd_squares(5)
!   integer :: s
!   s = 1
!   do i = 1,9,2
!       odd_squares(s) = i**2
!       s = s+1
!   end do
    odd_squares = [(i**2,i=1,9,2)]
    print *,odd_squares
end program main
\end{lstlisting}

我们可以对先前构造出来的一维数组进行变形(reshape)操作来获取多维数组,只需来个\ttt{a\_{}new = reshape(a\_{}old,s)}就成.\ttt{a\_{}new}和\ttt{a\_{}old}是两个数组,\ttt{s}是\ttt{a\_{}new}的形状(当然得是整型一维数组),\ttt{a\_{}new}的第$l$个元素和\ttt{a\_{}old}的第$l$个元素总是相同的\footnote{
不考虑赋值时的类型和种别转化的情况下.
}.整个变形操作说白了就是: 将数组\ttt{a\_{}old}中的元素复制到形状为\ttt{s}的数组\ttt{a\_{}new}中,并保证元素顺序不变,比如下面这样.
\begin{lstlisting}
program main
    implicit none
    integer :: one2four(2,2)            ! a11=1 a12=3
    one2four = reshape([1,2,3,4],[2,2]) ! a21=2 a22=4
end program main
\end{lstlisting}

\section{数组切片}

数组切片(slicing)是用一个数组得到另一个数组的操作.现假设有一个维数为\ttt{n}的数组\ttt{a},则\ttt{a(e1,...,en)}是另一个数组,其中\ttt{e1},\dots,\ttt{en}乃整型一维数组或整型标量.如何确定\ttt{a(e1,...,en)}?

引入\ttt{v1},\dots,\ttt{vn},保证若\ttt{ei}为数组则\ttt{vi}为\ttt{ei},若\ttt{ei}为标量则\ttt{vi}为\ttt{[ei]}.这样\ttt{a(v1,...,vn)}等于\ttt{b}.记\ttt{vi}为$(v_{i;1},\dots,v_{i;{l_i}})$,则$b_{{\iota_1},\dots,{\iota_n}}=a_{{v_{1;\iota_1}},\dots,{v_{n;\iota_n}}}$.

假设\ttt{e1},\dots,\ttt{en}中\ttt{ei1},\dots,\ttt{eim}\ ($i_1\!<\!\dots\!<\!i_m$)是长度为$l_{i_1},\dots,l_{i_m}$的数组,其他是标量,则可将\ttt{b}变形成形状为$(l_{i_1},\dots,l_{i_m})$的\ttt{c},\ttt{c}就是\ttt{a(e1,...,en)}.

我本来处心积虑地想再来几段话来把这切片讲得更明白些,然后我就放弃了,只好先来个示例给同志们做练习.我敢保证自己写的东东肯定是真实不虚的,但看来是很难理解记忆了.幸好非常复杂的数组切片一般是用不上的.如果老师敢考那些难死人的切片,我们就当即暴动$\!\text{\~{}}$
\begin{lstlisting}
program main      ! a000=1 a001=5   a100=2 a101=6
    implicit none ! a010=3 a011=7   a110=4 a111=8
    integer :: i,one2eight(0:1,0:1,0:1)
    integer :: result(1,4)
    one2eight = reshape([(i,i=1,8)],[2,2,2])
    result = one2eight(0,[0],[0,1,1,0]) ! The shape is [1,4].
    ! result X one2eight(0,0,[0,1,1,0]) ! The shape is [4]!
    print *,result
end program main
\end{lstlisting}

先前用向量下标(vector subscript)来切片,我们还可以用三元下标(triplet subscript)\footnote{
官方文档里用的是``subscript triplet''.
},以\ttt{p:q:r}代替\ttt{(i,i=p,q,r)},当然\ttt{:r}可以省略.这不是器规,是通用的.而且三元下标中\ttt{p}和\ttt{q}也可以省略(但是注意\ttt{p}和\ttt{q}之间的\ttt{:}不能省),\ttt{p}省略就等于那一维的下界,\ttt{q}省略就等于那一维的上界.这样的切片简单且比较常用(尤其是省略\ttt{:r}的时候),是要掌握的.比如我们可以方便地摘出$1$到$9$中的奇数.
\begin{lstlisting}
program main
    implicit none
    integer :: i
    integer :: singles(9),odds(5)
    singles = [(i,i=1,9)]
    odds = singles(::2) ! singles(1:9:2)
end program main
\end{lstlisting}
还有,如果\ttt{i1,...,in}都是整型标量,则\ttt{a(i1,...,in)}就是$a_{i_1\dots i_n}$,这更要掌握.问: \ttt{a(i1,...,in)}和\ttt{a(i1:i1,...,in:in)}有什么区别?

对数组切片,可以得到一个新数组,看起来可以对这个新数组再切片.然而这是不成的,原因在于对数组切片得到的新数组,其每一维的上下界其实都是不确定的\footnote{
虽然在介绍切片规则时看起来有确定的上下界,那只是为了说话方便.
},所以新数组中每个元素的下标都是不确定的,因此没法切片.同样地,由数组构造器得到的数组也是不能切片的.

\section{数组运算}

有了数组,总是要用数组来算些什么东西.现在我们可以在表达式中混用数组,标量和\ref{fortran_opration}节中的所有运算符.运算符的优先级,和之前是一样的,\ttt{+}和\ttt{-}之前如果什么也没有,也还是默认有个\ttt{0}.只需要知道,当运算符两边出现数组时会有什么结果,我们就能推理出任意表达式的结果了.这又分两种情况.
\begin{itemize}
\item 运算符两边都是数组.
\item 运算符一边是标量,另一边是数组.
\end{itemize}

如果运算符两边都是数组,我们首先必须保证这两个数组形状完全一致,绝对一致,这样这两个数组中的元素按所处的位置,自然就能一一对应.我们假设两个数组分别是$a$和$b$,并用符号$\star$表示一个运算符,现在我们要算$a\star b$.假设$a$和$b$的下界分别为$j_{a;1},\dots,j_{a;n}$和$j_{b;1},\dots,j_{b;n}$,则首先可以搞到另外两个数组$\alpha$和$\beta$,使得$\alpha_{i_1\dots i_n}=a_{(i_1+j_{a;1})\dots(i_n+j_{a;n})}$,$\beta_{i_1\dots i_n}=b_{(i_1+j_{b;1})\dots(i_n+j_{b;n})}$,然后可以弄出一个数组$c$,使得$c_{i_1\dots i_n}=\alpha_{i_1\dots i_n}\star\beta_{i_1\dots i_n}$,则$c$就是$a\star b$.简单来说就是对$a$和$b$中两两对应的元素进行$\star$运算,得到新数组.示例如下.
\begin{lstlisting}
program main      ! 1+5=6 3+7=10
    implicit none ! 2+6=8 4+8=12
    integer :: one2four(2,2),five2eight(2,2,1)
    one2four = reshape([1,2,3,4],[2,2])
    five2eight = reshape([5,6,7,8],[2,2,1])
    print *,one2four+reshape(five2eight,[2,2])
    ! print *,one2four+five2eight (X)
end program main
\end{lstlisting}

如果运算符一边是标量,另一边是数组,不妨设数组为$a$,标量为$b$,仍设运算符为$\star$.此时若$a$的形状为$\vec{s}$,则可以另搞一个形状为$\vec{s}$的数组$\tilde{b}$,使得$\tilde{b}$中任意元素都是$b$,然后就有$a\star b=a\star\tilde{b}$,$b\star a=\tilde{b}\star a$.示例如下.
\begin{lstlisting}
program main      ! 1+9=10 3+9=12
    implicit none ! 2+9=11 4+9=13
    integer :: one2four(2,2)
    one2four = reshape([1,2,3,4],[2,2])
    print *,one2four+9
end program main
\end{lstlisting}

即使表达式的结果是一个数组,也不能对其切片,因为此时数组的上下界依然是不确定的.比如,有两个一维数组,一个下界是$1$,一个下界是$0$,把这俩加起来,得到的数组上下界应该是多少?不好规定.有些情况下,比如让两个下界都是$1$的一维数组相加,看起来数组上下界是好规定的,然而若真来个规定,造编译器的人就得处心积虑地要让编译器能够区分这两种情形,他们会很不开心,很不快乐,所以统统规定上下界不确定是好的.

\section{数组赋值}\label{fortran_array_assignment}

数组赋值可分两种,一种``\ttt{=}''右侧是数组,另一种``\ttt{=}''右侧是标量.

如果``\ttt{=}''右侧是数组,则俩玩楞形状必须一样滴.假设我们要令\ttt{a=b},并设$n$维数组$a$和$b$的下界分别为$j_{a;1},\dots,j_{a;n}$和$j_{b;1},\dots,j_{b;n}$,则$a$和$b$的元素自然能一一对应.首先我们要搞一个$\tilde{b}$,使得$\tilde{b}_{(i_1-j_{b;1}+j_{a;1})\dots(i_n-j_{b;n}+j_{a;n})}=b_{i_1\dots i_n}$,然后令$a$等于$\tilde{b}$即可.简单来说就是令$a$和$b$中两两对应的元素相等.这里$b$的上下界可能是不确定的,但赋值给$a$后,$a$先前声明过,所以其上下界一定是确定的.

如果``\ttt{=}''右侧是标量,比如数组是$a$,标量是$b$,则先把$b$变成和$a$形状相同的数组$c$,使得$c$中任意元素都是$b$,然后令\ttt{a=c}即可.

对于数组,我们还可以用一些特殊东东来赋值,比如forall,where和do concurrent,不过这些东东我自己貌似会用,却研究不出它们的明确规则.我计划在第\ref{fortran_parallel_conpute}章介绍它们.

\chapter{编译器}\label{fortran_compiler}
\def\r{${}^\text{\textregistered}$}

\begin{table}[!htbp]
    \centering
    \begin{tabular}{|c|c|c|}
        \hline
        编译器&Ifx (+VS)&Gfortran (+VS Code)\\
        \hline
        类型&专有软件&自由软件\\
        \hline
        总空间占用&约 5.2G&约 780M\\
        \hline
        语法支持&略多于 Gfortran, 较宽松&略少于 Ifx, 较严格\\
        \hline
        自动纠错和自动补全&无&有\\
        \hline
        编译提示信息&次于 Gfortran&优于 Ifx\\
        \hline
        平均运行速度&快于 Gfortran&慢于 Ifx\\
        \hline
    \end{tabular}
    \caption{Ifx (+VS) 与 Gfortran (+VS Code) 的对比}
\end{table}

\section[Intel\r{} Fortran Compiler]{Intel\r{} Fortran Compiler (Ifx)}

\subsection{安装}

Windows 系统安装方式如下.
\begin{enumerate}
    \item 安装 \href{https://visualstudio.microsoft.com/zh-hans/thank-you-downloading-visual-studio/?sku=Community&channel=Release&version=VS2022&source=VSLandingPage&cid=2030&passive=false}
    {Visual Studio Community 2022}. 安装时可选择语言为中文.
    \item ``工作负载'' 选择 ``使用 C++ 的桌面开发''.
    \item 安装 \href{https://registrationcenter-download.intel.com/akdlm/IRC_NAS/7feb5647-59dd-420d-8753-345d31e177dc/w_fortran-compiler_p_2024.2.0.424.exe}{Intel\r{} Fortran Compiler}.
    \item 找到开始菜单内 Intel oneAPI 2023 文件夹中的 Intel oneAPI command prompt for Intel 64 for Visual Studio 2022 和 Intel oneAPI command prompt for IA32 for Visual Studio 2022, 右键选择 ``更多'' $\rightarrow$ ``打开文件位置'', 可以找这两个东东的快捷方式. 然后右键这两个快捷方式, 选择 ``属性'', 把 ``起始路径'' 改成自己最常访问的路径 (比如桌面的路径), 然后点 ``应用'', 点 ``继续'', 点 ``确定''.\label{to_desktop}
\end{enumerate}

\subsection{使用}\label{use_ifx}

\subsubsection{使用 CLI}

现在俺们通过一个实例来掌握 CLI 的用法! 我们先在任意一个文件夹里新建两个空白文本文档 (默认的文件名应该是 ``\textsf{新建}\texttt{\ }\textsf{文本文档}\texttt{.txt}''), 分别把文件名改为 \texttt{main.f90} 和 \texttt{helloworld.f90} (注意要把拓展名 \texttt{.txt} 也给改掉, 不要改成 \texttt{main.f90.txt} 和 \texttt{helloworld.f90.txt} 哟).

打开 \texttt{main.f90}, 写入下面的内容并保存.
\begin{lstlisting}
program main
    implicit none
    call helloworld()
end program main
\end{lstlisting}

打开 \texttt{helloworld.f90}, 写入下面的内容并保存.
\begin{lstlisting}
subroutine helloworld()
    implicit none
    print *, 'Hello, world!'
end subroutine helloworld
\end{lstlisting}

然后打开开始菜单内 Intel oneAPI 2023 文件夹中的 Intel oneAPI command prompt for Intel 64 for Visual Studio 2022 (32 位系统请打开 Intel oneAPI command prompt for IA32 for Visual Studio 2022), 这样会蹦出一个可怕的黑框框.

接下来俺们要 ``\texttt{cd}'' 到 \texttt{main.f90} 和 \texttt{helloworld.f90} 所在的文件夹去. 安安把 \texttt{main.f90} 和 \texttt{helloworld.f90} 放在桌面, 安安桌面的路径是 \texttt{C:\bs{}Users\bs{}GasinAn\bs{}Desktop}, 所以安安要在黑框框里输入
\begin{verbatim}
cd /D C:\Users\GasinAn\Desktop
\end{verbatim}
然后按回车, 这样黑框框里最后一行 ``\texttt{>}'' 的左边就会变成桌面的路径. 不过安安其实不用干这一步, 因为在安装时, 第 \ref{to_desktop} 步已经设置好打开黑框框时默认 ``\texttt{cd}'' 到桌面了. 假如 \texttt{main.f90} 和 \texttt{helloworld.f90} 不在桌面而在 \texttt{D:\bs{}Documents} 文件夹, 安安就得输入 \texttt{cd /D D:\bs{}Documents} 了.

然后俺们要编译 \texttt{main.f90} 和 \texttt{helloworld.f90}, 在黑框框里输入
\begin{verbatim}
ifx *.f* /o a.exe
\end{verbatim}
然后按回车, 这样桌面会多出个 \texttt{a.exe} 文件来\footnote{还会多出 \texttt{main.obj} 和 \texttt{helloworld.obj}, 这些拓展名是 \texttt{.obj} 的文件初学 Fortran 的小盆友们请无视, 删了也没关系.}. 这步俺们干的是 ``把文件名 (连同拓展名) 里带 `\texttt{.f}' 的文件编译成 \texttt{a.exe}'', 万一 \texttt{main.f90} 和 \texttt{helloworld.f90} 所在的文件夹里还有其他文件名里带 ``\texttt{.f}'' 的文件, 就会坏事, 请同学们自行请这些文件暂时搬家到别的地方去.

最后我们要运行 \texttt{a.exe}, 在黑框框里输入
\begin{verbatim}
a.exe
\end{verbatim}
然后按回车, 看到黑框框里蹦出 \texttt{Hello, world!} 就大成功!

\subsubsection{使用 GUI}

打开 Visual Studio, 点 ``创建新项目''.

然后搜索 ``Empty Project'', 找到下面标有 ``Fortran'', ``Windows'', ``控制台'' 的 ``Empty Project'', 选择之, 然后点击 ``下一步''.

然后把 ``项目名称'' 改成 ``HelloWorld'', ``位置'' 选择自己喜欢的文件夹 (比如桌面) 的路径 (下面用 \texttt{[dir]} 表示这个路径), 点击 ``创建''.

然后就出现了编辑界面, 并且 \texttt{[dir]} 内多出了一个 \texttt{HelloWorld} 文件夹.

默认使用的是 Ifort\footnote{Intel\r{} Fortran Compiler Classic,这玩意儿 Intel\r{} 不想玩儿了,要用 Ifx 来代替.}.欲用 Ifx, 右键右边 ``解决方案资源管理器'' 里的 ``Console1 (Ifx)'',点最下面的 ``属性'', 左边选 ``配置属性'' $\rightarrow$ ``General'' (应已自动选上), 点 ``Use Compiler'' 右边的 ``Ifx Intel$\,\text{\textregistered}$ Fortran Compiler Classic'', 点最右边的带向下标志的按钮, 改选成``IFX Intel$\,\text{\textregistered}$ Fortran Compiler'', 点 ``应用'', 点 ``确定'',看到右边 ``Console1 (Ifx)'' 变成 ``Console1 (IFX)'' 即成功.

64 位系统, 若上面 Debug 后是 x86, 则可能需要将上面 Debug 后的 x86 改成 x64 (如不需要, 也最好改改). 点 x86 右边的向下箭头可以改, 若没有 x64, 可以点 ``配置管理器\dots'' 后尝试把 x64 调出来.

右击右边 ``解决方案资源管理器'' 中的 ``Source Files'', 选择 ``添加'' $\rightarrow$ ``现有项\dots'', 然后把之前的 \texttt{main.f90} 和 \texttt{helloworld.f90} 添加进来. 然后双击文件名即可打开文件.

点击上面的 ``调试'', 然后点 ``开始执行 (不调试)'', 看到下面框框里蹦出 \texttt{Hello, world!} 就大成功!

右击右边 ``解决方案资源管理器'' 中的 ``Source Files'', 选择 ``添加'' $\rightarrow$ ``新建项\dots'', 可以新建文件, 默认在 \texttt{HelloWorld} 文件夹里的 \texttt{HelloWorld} 文件夹里. 右击任意一个文件的文件名后, 可以点击 ``删除'' 或 ``重命名'' 进行删除或重命名操作.

关掉 Visual Studio, 重新打开, 左边多出了个 HelloWorld.sln, 点它就能回到编辑界面了.

\section[GNU Fortran Compiler]{GNU Fortran Compiler (Gfortran)}

\subsection{安装}

Windows 系统安装方式如下.\footnote{
    以下是直接安装 MinGW-w64 来安装 Gfortran 的, 但如果有 Python, 也许能直接通过 Pip (或 Conda) 来安装 Gfortran (或 MinGW-w64, MSYS2, \dots)? 我也不知\dots
}
\begin{enumerate}
    \item 访问 \href{https://github.com/niXman/mingw-builds-binaries/releases}
    {MinGW-Builds-binaries releases}, 选择最新的 release 进入.
    \item 选带 posix 的; 64 位系统选带 x86\_{}64 的, 32 位系统选带 i686 的; Win 10 及以上选带 ucrt 的, Win 10 以下选带 msvcrt 的. 下载并解压.
    \item 把解压出来的名为 \texttt{mingw64} 的文件夹剪切到随便哪个目录. 暂称粘贴到的目录为 \texttt{[dir]}.
    \item 在系统环境变量 \texttt{Path} 中加入 \texttt{[dir]\bs{}mingw64\bs{}bin}.比如, 如果刚才粘贴到 \texttt{C:\bs{}Program Files}, 就加入 \texttt{C:\bs{}Program Files\bs{}mingw64\bs{}bin}.
    \item 下载 \href{https://code.visualstudio.com/sha/download?build=stable&os=win32-x64-user}{Visual Studio Code} 并安装.
    \item 打开 Visual Studio Code, 点击左边四个正方形飞出一个的图标, 搜索 C/C++,Modern Fortran 和 Code Runner 并安装.
    \item[] 如果已经装了 Python, 装 Modern Fortran 前先用 Pip\footnote{有 Conda 当然用 Conda 啦!} 装 fortls .
    \item 按 Ctrl+Shift+P, 然后选择 ``Preferences: Open Settings (JSON)'', 打开名为 \texttt{settings.json} 的 JSON 文件.
    \item 在 \texttt{settings.json} 里加入下面这些键值对.\label{add_key_value}
    \begin{verbatim}
"code-runner.executorMap": {
    "FortranFreeForm":
    "cd $dir; gfortran *.f*; if($?){.\a.exe}",
    "FortranFixedForm":
    "cd $dir; gfortran *.f*; if($?){.\a.exe}"
},
"code-runner.runInTerminal": true,
"code-runner.saveAllFilesBeforeRun": true
    \end{verbatim}
\end{enumerate}
最后第 \ref{add_key_value} 步还需解释, 因为牵涉到 JSON 文件的语法. JSON 文件里的内容应该满足下面的形式.
\begin{verbatim}
{
  key_1: value_1,
  ...
  key_n: value_n
}
\end{verbatim}
每一个形如 \texttt{key\_{}i: value\_{}i} 的东东称作一个键值对, 任意两个键值对之间用逗号隔开. 所以, 如果一开始 \texttt{settings.json} 里头空空如也, 则加入键值对后可能长这样.
\begin{verbatim}
{
  "code-runner.executorMap": {
    "FortranFreeForm":
    "cd $dir; gfortran *.f*; if($?){.\a.exe}",
    "FortranFixedForm":
    "cd $dir; gfortran *.f*; if($?){.\a.exe}"
  },
  "code-runner.runInTerminal": true,
  "code-runner.saveAllFilesBeforeRun": true
}
\end{verbatim}
如果一开始 \texttt{settings.json} 里长这样,
\begin{verbatim}
{
  "editor.wordWrap": "wordWrapColumn",
  "editor.wordWrapColumn": 80,
  "workbench.colorTheme": "Red"
}
\end{verbatim}
则加入键值对后可能长这样.
\begin{verbatim}
{
  "editor.wordWrap": "wordWrapColumn",
  "editor.wordWrapColumn": 80,
  "workbench.colorTheme": "Red",
  "code-runner.executorMap": {
    "FortranFreeForm":
    "cd $dir; gfortran *.f*; if($?){.\a.exe}",
    "FortranFixedForm":
    "cd $dir; gfortran *.f*; if($?){.\a.exe}"
  },
  "code-runner.runInTerminal": true,
  "code-runner.saveAllFilesBeforeRun": true
}
\end{verbatim}
注意第 4 行最后要多加一个逗号来隔开键值对.

\subsection{使用}\label{use_gfortran}

\subsubsection{使用 CLI}

Gfortran 的用法和之前 \ref{use_ifx} 小节中 Ifx 的用法是很像的.

首先俺们需要打开 Powershell, 可以按 Win+R, 输入 ``powershell'' 后点 ``确定'', 也可以在 Visual Studio Code 中, 点上面 ``Terminal'' 后点 ``New Terminal''.

然后 ``\texttt{cd}'' 那步, 输入的东东要去掉 ``\texttt{/D}'', 示例如下.
\begin{verbatim}
cd C:\Users\GasinAn\Desktop
\end{verbatim}
这里我们没有设置好打开 Powershell 时默认 ``\texttt{cd}'' 到桌面, 所以不好偷懒. 同志们如果想设置来偷懒, 可以仿照 Ifx 安装第 \ref{to_desktop} 步, 对 Powershell 也同样来波操作.

之后编译那步, 输入的东东如下.
\begin{verbatim}
gfortran *.f*
\end{verbatim}

最后运行那步是一样的.

\subsubsection{使用 GUI}

用 Visual Studio Code 打开 \texttt{main.f90}\footnote{之后同学们学 \ref{program_unit} 节,会明白 \texttt{main.f90} 里写的是``主程序'',所以打开它.}, 然后点右上方的白色三角儿即可大功告成!

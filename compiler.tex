\chapter{编译器}\label{fortran_compiler}
\def\r{${}^\text{\textregistered}$}

\begin{table}[!htbp]
    \centering
    \begin{tabular}{|c|c|c|}
        \hline
        编译器&Ifort (+VS)&Gfortran (+VS Code)\\
        \hline
        类型&专有软件&自由软件\\
        \hline
        总空间占用&约5.2G&约780M\\
        \hline
        语法支持&略多于Gfortran, 较宽松&略少于Ifort, 较严格\\
        \hline
        自动纠错和自动补全&无&有\\
        \hline
        编译提示信息&次于Gfortran&优于Ifort\\
        \hline
        平均运行速度&快于Gfortran&慢于Ifort\\
        \hline
    \end{tabular}
    \caption{Ifort (+VS)与Gfortran (+VS Code)的对比}
\end{table}

\section[Intel\r{} Fortran Compiler]{Intel\r{} Fortran Compiler (Ifort)}

\subsection{安装}

Windows系统安装方式如下.
\begin{enumerate}
    \item 安装~\href{https://visualstudio.microsoft.com/zh-hans/thank-you-downloading-visual-studio/?sku=Community&rel=16}
    {Visual Studio Community 2019}. 安装时可选择语言为中文.
    \item ``工作负载''选择``使用C++的桌面开发''.
    \item 安装~\href{https://registrationcenter-download.intel.com/akdlm/irc_nas/18412/w_fortran-compiler_p_2022.0.0.77_offline.exe}
    {Intel\r{} Fortran Compiler}.
    \item 找到开始菜单内Intel oneAPI 2022文件夹中的Intel oneAPI command prompt for Intel 64 for Visual Studio 2019和Intel oneAPI command prompt for IA32 for Visual Studio 2019, 右键选择``更多''$\rightarrow$``打开文件位置'', 可以找这两个东东的快捷方式. 然后右键这两个快捷方式,选择``属性'', 把``起始路径''改成自己最常访问的路径.
\end{enumerate}

\subsection{使用}\label{use_ifort}

\subsubsection{使用CLI}

Windows系统中,打开开始菜单内Intel oneAPI 2021中的Intel oneAPI command prompt for Intel 64 for Visual Studio 2019 (或Intel oneAPI command prompt for IA32 for Visual Studio 2019). 然后可以使用cmd的命令(比如\verb|cd|之类的).

\paragraph{示例1}
新建一个名为\verb|main.f90|的文件,写入下面的内容并保存.
\begin{verbatim}
program main
    implicit none
    print *, 'Hello, world!'
end program main
\end{verbatim}

然后\verb|cd|到\verb|main.f90|所在的目录后执行下面这条命令.
\begin{verbatim}
ifort main.f90
\end{verbatim}
就把\verb|main.f90|编译好啦,生成了文件\verb|main.obj|和\verb|main.exe|.

执行下面这条命令.
\begin{verbatim}
main.exe
\end{verbatim}
``\verb|Hello, world!|''被打出来啦!

\paragraph{示例2}
删掉\verb|main.obj|和\verb|main.exe|, 在\verb|main.f90|所在目录再新建一个名为\\\verb|helloworld.f90|的文件,写入下面的内容并保存.
\begin{verbatim}
subroutine helloworld
    implicit none
    print *, 'Hello, world!'
end subroutine helloworld
\end{verbatim}

把\verb|main.f90|修改成下面这样.
\begin{verbatim}
program main
    implicit none
    call helloworld
end program main
\end{verbatim}

执行下面这条命令.
\begin{verbatim}
ifort main.f90 helloworld.f90
\end{verbatim}
生成了\verb|main.obj|,~\verb|helloworld.obj|和\verb|main.exe|.

然后再次执行下面的命令.
\begin{verbatim}
main.exe
\end{verbatim}
``\verb|Hello, world!|''被打出来啦!

\paragraph{示例3}
以后搞一些大事情时,~\verb|ifort|后要跟一大堆文件.

一个个打文件名显然要死人.这时可以使用通配符.

删掉\verb|main.obj|,~\verb|helloworld.obj|和\verb|main.exe|.执行命令.
\begin{verbatim}
ifort *.f90
\end{verbatim}
就相当于\verb|ifort|后跟了当前目录所有最后是\verb|.f90|的文件的文件名.但注意,这回生成的不是\verb|main.exe|,而是\verb|helloworld.exe|,字母表中h可是排在m前面呢.

执行命令.
\begin{verbatim}
helloworld.exe
\end{verbatim}
``\verb|Hello, world!|''被打出来啦!

\paragraph{示例4}
文件太多时,使用通配符后可能不清楚最后\verb|.exe|文件的文件名是什么.这时可以干脆直接指定文件名.

删掉\verb|main.obj|,~\verb|helloworld.obj|和\verb|helloworld.exe|.执行命令.
\begin{verbatim}
ifort *.f90 -o a
\end{verbatim}
这回生成的不是\verb|helloworld.exe|,而是\verb|a.exe|.

执行命令.
\begin{verbatim}
a.exe
\end{verbatim}
``\verb|Hello, world!|''被打出来啦!

\paragraph{示例5}
删掉\verb|main.obj|,~\verb|helloworld.obj|和\verb|a.exe|.执行命令.
\begin{verbatim}
ifort -c helloworld.f90
\end{verbatim}
这回只生成了文件\verb|helloworld.obj|

执行命令.
\begin{verbatim}
ifort main.f90 helloworld.obj
\end{verbatim}
生成了\verb|main.obj|和\verb|main.exe|.

执行命令.
\begin{verbatim}
main.exe
\end{verbatim}
``\verb|Hello, world!|''被打出来啦!

Ifort的工作流程是,先把每个源文件都变成对应的\verb|.obj|文件,再把它们合起来后变成\verb|.exe|文件.遇到一些一万年也不会改,而且会反复使用的文件(比如轮子\footnote{
    这里玩了个``重新制造轮子''的梗.
}里的文件),我们就可以先把它们统统变成\verb|.obj|文件,然后反复用,以节省时间.

\subsubsection{使用GUI}

打开Visual Studio, 点``创建新项目''.

然后搜索``Empty Project'', 找到下面标有``Fortran'',``Windows'',``控制台''的``Empty Project'',选择之,然后点击``下一步''.

然后把``项目名称''改成HelloWorld,``位置''选择自己喜欢的路径(下面用\verb|[dir]|表示这个路径),点击``创建''.

然后就出现了编辑界面,并且\verb|[dir]|内多出了一个\verb|HelloWorld|文件夹.

64位系统,可能需要将上面Debug后的x86改成x64,如不需要,也最好改改.点x86右边的向下箭头可以改,若没有x64,可以点``配置管理器\dots''后尝试把x64调出来.

右击右边``解决方案资源管理器''中的``Source Files'',选择``添加''$\rightarrow$``现有项\dots'',然后把之前的\verb|main.f90|和\verb|helloworld.f90|添加进来.然后双击文件名即可打开文件.

点击上面的``调试'',然后点``开始执行(不调试)''.

``\verb|Hello, world!|''被打出来啦!

右击右边``解决方案资源管理器''中的``Source Files'',选择``添加''$\rightarrow$``新建项\dots'',可以新建文件,默认在\verb|HelloWorld|文件夹里的\verb|HelloWorld|文件夹里.右击任意一个文件的文件名后,可以点击``删除''或``重命名''进行删除或重命名操作.

关掉Visual Studio,重新打开,左边多出了个HelloWorld.sln,点它就能回到编辑界面了.

\section[GNU Fortran Compiler]{GNU Fortran Compiler (Gfortran)}

\subsection{安装}

Windows系统安装方式如下.\footnote{
    以下是直接安装MinGW-w64来安装Gfortran的, 但如果有Python, 也许能直接通过Pip~(或Conda)来安装Gfortran~(或MinGW-w64, MSYS2, \dots)? 我也不知\dots
}
\begin{enumerate}
    \item 下载~\href{https://sourceforge.net/projects/mingw-w64/files/Toolchains%20targetting%20Win32/Personal%20Builds/mingw-builds/installer/mingw-w64-install.exe/download}
    {MinGW-w64}\footnote{
      怎么更新MinGW-w64是个问题. 也许装MSYS2更好?
    }安装包并打开.
    \item 64位系统, 先选择``Architecture''为``x86\_64'', 后选择``Threads''为``win32'', ``Exceptions''为``seh'', 然后安装.\\32位系统, 先选择``Architecture''为``i686'', 后选择``Threads''为``win32'', ``Exceptions''为``dwarf'', 然后安装.
    \item 在系统环境变量\verb|Path|中加入\verb|[dir]\mingw64\bin|.\\其中\verb|[dir]|要替换为MinGW-w64的安装目录,在这个目录下应该能见到一个名为\verb|mingw-w64.bat|的文件.
    \item 下载~\href{https://code.visualstudio.com/sha/download?build=stable&os=win32-x64-user}
    {Visual Studio Code}~并安装.
    \item 打开Visual Studio Code, 点击左边四个正方形飞出一个的图标,搜索C/C++,  Modern Fortran和Code Runner并安装.
    \item[] 如果已经装了Python, 装Modern Fortran前先用pip\footnote{有conda当然用conda啦$\!\text{\~{}}$}装fortls.
    \item 按Ctrl+Shift+P,然后选择``Preferences: Open Settings (JSON)'',打开名为\verb|settings.json|的JSON文件.
    \item 在\verb|settings.json|里加入下面这些键值对.
    \begin{verbatim}
"code-runner.executorMap": {
    "FortranFreeForm":
    "[cmd]",
    "fortran_fixed-form":
    "[cmd]"
},
"code-runner.runInTerminal": true,
"code-runner.saveAllFilesBeforeRun": true,
"fortran.includePaths": [
  "[dir]mingw64\\bin"
]
    \end{verbatim}
\end{enumerate}

最后一步其实和在VS Code的设置里点点点是一样滴, 我其实就是在设置里凭感觉点点点后把\verb|settings.json|里的东东弄出来了, 如果玩代码的时候感觉不舒服, 自己解决, 我可不管. 注意
\begin{itemize}
    \item \verb|[cmd]|要替换成编译并执行程序的整套命令.可以参考\ref{use_gfortran}小节, 按自己的需求替换,也可以戳\href{https://zhuanlan.zhihu.com/p/362328064}{这里}搬运现成的.
    \item \verb|[dir]|要替换为MinGW-w64的安装目录. 注意要多加一个\verb|\|来转义\verb|\|. (若连``转义''也不知何物, 也可看下面的例子)
\end{itemize}

还有加入键值对的时候要符合JSON的规则. 举两个例子.
如果一开始\verb|settings.json|里是空的,加入键值对后可能长这样.
\begin{verbatim}
{
  "code-runner.executorMap": {
    "FortranFreeForm":
    "cd $dir; gfortran *.f*; if($?){.\\a.exe}",
    "fortran_fixed-form":
    "cd $dir; gfortran *.f*; if($?){.\\a.exe}"
  },
  "code-runner.runInTerminal": true,
  "code-runner.saveAllFilesBeforeRun": true,

  "fortran.includePaths": [
    "C:\\Program Files\\mingw-w64\\mingw64\\bin"
  ]
}
\end{verbatim}
如果一开始\verb|settings.json|长这样,
\begin{verbatim}
{
  "editor.wordWrap": "wordWrapColumn",
  "editor.wordWrapColumn": 80,
  "workbench.colorTheme": "Red"
}
\end{verbatim}
加入键值对后就可能长这样.
\begin{verbatim}
{
  "editor.wordWrap": "wordWrapColumn",
  "editor.wordWrapColumn": 80,
  "code-runner.executorMap": {
    "FortranFreeForm":
    "cd $dir; gfortran *.f*; if($?){.\\a.exe}",
    "fortran_fixed-form":
    "cd $dir; gfortran *.f*; if($?){.\\a.exe}"
  },
  "code-runner.runInTerminal": true,
  "code-runner.saveAllFilesBeforeRun": true,
  "fortran.includePaths": [
    "C:\\Program Files\\mingw-w64\\mingw64\\bin"
  ],
  "workbench.colorTheme": "Red"
}
\end{verbatim}
注意倒二行最后多了一个逗号. 键值对间要用逗号隔开.

~

~

\subsection{使用}\label{use_gfortran}

\subsubsection{使用CLI}
Gfortran的用法和\ref{use_ifort}小节中Ifort的用法非常像,不同点是:
\begin{itemize}
  \item 使用\verb|gfortran|命令代替\verb|ifort|命令;
  \item 中间过程生成的是\verb|.o|文件,不是\verb|.obj|文件;
  \item 默认生成文件\verb|a.exe|,且不会生成\verb|.o|文件.
\end{itemize}

直接打开命令行(cmd或powershell),即可使用\verb|gfortran|命令.

在VS Code中, 可以直接点上面``Terminal''后点``New Terminal''来调出命令行.默认调出的是powershell.若想让默认调出的命令行是cmd,可在\verb|settings.json|里加入下面这个键值对.
\begin{verbatim}
"terminal.integrated.shell.windows":
"C:\\Windows\\System32\\cmd.exe"
\end{verbatim}

\subsubsection{使用GUI}
点击右上方的白色三角儿即可编译并执行! 除非之前\verb|[cmd]|设错啦$\!\text{\~{}}$

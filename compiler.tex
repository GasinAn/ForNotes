\chapter{编译器}\label{fortran_compiler}
\def\r{${}^\text{\textregistered}$}

\begin{table}[!htbp]
    \centering
    \begin{tabular}{|c|c|c|}
        \hline
        编译器&Ifort (+VS)&Gfortran (+VS Code)\\
        \hline
        类型&专有软件&自由软件\\
        \hline
        总空间占用&约5.2G&约780M\\
        \hline
        语法支持&略多于Gfortran, 较宽松&略少于Ifort, 较严格\\
        \hline
        自动纠错和自动补全&无&有\\
        \hline
        编译提示信息&次于Gfortran&优于Ifort\\
        \hline
        平均运行速度&快于Gfortran&慢于Ifort\\
        \hline
    \end{tabular}
    \caption{Ifort (+VS)与Gfortran (+VS Code)的对比}
\end{table}

\section[Intel\r{} Fortran Compiler]{Intel\r{} Fortran Compiler (Ifort)}

\subsection{安装}

Windows系统安装方式如下.
\begin{enumerate}
    \item 安装~\href{https://visualstudio.microsoft.com/zh-hans/thank-you-downloading-visual-studio/?sku=Community&channel=Release&version=VS2022&source=VSLandingPage&cid=2030&passive=false}
    {Visual Studio Community 2022}. 安装时可选择语言为中文.
    \item ``工作负载''选择``使用C++的桌面开发''.
    \item 安装~\href{https://registrationcenter-download.intel.com/akdlm/IRC_NAS/1720594b-b12c-4aca-b7fb-a7d317bac5cb/w_fortran-compiler_p_2023.2.1.7.exe}
    {Intel\r{} Fortran Compiler}.
    \item 找到开始菜单内Intel oneAPI 2023文件夹中的Intel oneAPI command prompt for Intel 64 for Visual Studio 2022和Intel oneAPI command prompt for IA32 for Visual Studio 2022, 右键选择``更多''$\rightarrow$``打开文件位置'', 可以找这两个东东的快捷方式. 然后右键这两个快捷方式, 选择``属性'', 把``起始路径''改成自己最常访问的路径(比如桌面的路径), 然后点``应用'', 点``继续'', 点``确定''.
\end{enumerate}

\subsection{使用}\label{use_ifort}

\subsubsection{使用CLI}

Windows系统中, 打开开始菜单内Intel oneAPI 2023文件夹中的Intel oneAPI command prompt for Intel 64 for Visual Studio 2022 (或Intel oneAPI command prompt for IA32 for Visual Studio 2022). 然后可以使用cmd的命令(比如\texttt{cd}之类的).

\paragraph{示例1}
新建一个名为\texttt{main.f90}的文件, 写入下面的内容并保存.
\begin{lstlisting}
program main
    implicit none
    print *, 'Hello, world!'
end program main
\end{lstlisting}

然后\texttt{cd}到\texttt{main.f90}所在的目录后执行下面这条命令\footnote{这里\texttt{ifort}对应的其实是Intel\r{} Fortran Compiler Classic, 想用真Intel\r{} Fortran Compiler, 请换成\texttt{ifx}. 本书所有示例程序都是用\texttt{ifort}测试的, 懒得改了, 大家凑合看吧!}.
\begin{lstlisting}
ifort main.f90
\end{lstlisting}
就把\texttt{main.f90}编译好啦, 生成了文件\texttt{main.obj}和\texttt{main.exe}.

执行下面这条命令.
\begin{lstlisting}
main.exe
\end{lstlisting}
``\texttt{Hello, world!}''被打出来啦!

\paragraph{示例2}
删掉\texttt{main.obj}和\texttt{main.exe}, 在\texttt{main.f90}所在目录再新建一个名为\\\texttt{helloworld.f90}的文件, 写入下面的内容并保存.
\begin{lstlisting}
subroutine helloworld
    implicit none
    print *, 'Hello, world!'
end subroutine helloworld
\end{lstlisting}

把\texttt{main.f90}修改成下面这样.
\begin{lstlisting}
program main
    implicit none
    call helloworld
end program main
\end{lstlisting}

执行下面这条命令.
\begin{lstlisting}
ifort main.f90 helloworld.f90
\end{lstlisting}
生成了\texttt{main.obj},~\texttt{helloworld.obj}和\texttt{main.exe}.

然后再次执行下面的命令.
\begin{lstlisting}
main.exe
\end{lstlisting}
``\texttt{Hello, world!}''被打出来啦!

\paragraph{示例3}
以后搞一些大事情时,~\texttt{ifort}后要跟一大堆文件.

一个个打文件名显然要死人.这时可以使用通配符.

删掉\texttt{main.obj},~\texttt{helloworld.obj}和\texttt{main.exe}.执行命令.
\begin{lstlisting}
ifort *.f90
\end{lstlisting}
就相当于\texttt{ifort}后跟了当前目录所有最后是\texttt{.f90}的文件的文件名. 但注意, 这回生成的不是\texttt{main.exe}, 而是\texttt{helloworld.exe}, 字母表中h可是排在m前面呢.

执行命令.
\begin{lstlisting}
helloworld.exe
\end{lstlisting}
``\texttt{Hello, world!}''被打出来啦!

\paragraph{示例4}
文件太多时, 使用通配符后可能不清楚最后\texttt{.exe}文件的文件名是什么. 这时可以干脆直接指定文件名.

删掉\texttt{main.obj},~\texttt{helloworld.obj}和\texttt{helloworld.exe}. 执行命令.
\begin{lstlisting}
ifort *.f90 -o a
\end{lstlisting}
这回生成的不是\texttt{helloworld.exe}, 而是\texttt{a.exe}.

执行命令.
\begin{lstlisting}
a.exe
\end{lstlisting}
``\texttt{Hello, world!}''被打出来啦!

\paragraph{示例5}
删掉\texttt{main.obj},~\texttt{helloworld.obj}和\texttt{a.exe}. 执行命令.
\begin{lstlisting}
ifort -c helloworld.f90
\end{lstlisting}
这回只生成了文件\texttt{helloworld.obj}.

执行命令.
\begin{lstlisting}
ifort main.f90 helloworld.obj
\end{lstlisting}
生成了\texttt{main.obj}和\texttt{main.exe}.

执行命令.
\begin{lstlisting}
main.exe
\end{lstlisting}
``\texttt{Hello, world!}''被打出来啦!

Ifort的工作流程是, 先把每个源文件都变成对应的\texttt{.obj}文件, 再把它们合起来后变成\texttt{.exe}文件. 遇到一些一万年也不会改, 而且会反复使用的文件(比如轮子\footnote{
    这里玩了个``重新制造轮子''的梗.
}里的文件), 我们就可以先把它们统统变成\texttt{.obj}文件, 然后反复用, 以节省时间.

\subsubsection{使用GUI}

打开Visual Studio, 点``创建新项目''.

然后搜索``Empty Project'', 找到下面标有``Fortran'', ``Windows'', ``控制台''的``Empty Project'', 选择之, 然后点击``下一步''.

然后把``项目名称''改成HelloWorld, ``位置''选择自己喜欢的路径(下面用\texttt{[dir]}表示这个路径), 点击``创建''.

然后就出现了编辑界面, 并且\texttt{[dir]}内多出了一个\texttt{HelloWorld}文件夹.

默认使用的是Ifort. 欲用Ifx, 右键右边``解决方案资源管理器''里的``Console1 (IFORT)'', 点最下面的``属性'', 左边选``配置属性''$\rightarrow$``General'' (应已自动选上), 点``Use Compiler''右边的``IFORT Intel$\,\text{\textregistered}$ Fortran Compiler Classic'', 点最右边的带向下标志的按钮, 改选成`IFX Intel$\,\text{\textregistered}$ Fortran Compiler'', 点``应用'', 点``确定'', 看到右边``Console1 (IFORT)''变成``Console1 (IFX)''即成功.

64位系统, 若上面Debug后是x86, 则可能需要将上面Debug后的x86改成x64 (如不需要, 也最好改改). 点x86右边的向下箭头可以改, 若没有x64, 可以点``配置管理器\dots''后尝试把x64调出来.

右击右边``解决方案资源管理器''中的``Source Files'', 选择``添加''$\rightarrow$``现有项\dots'', 然后把之前的\texttt{main.f90}和\texttt{helloworld.f90}添加进来. 然后双击文件名即可打开文件.

点击上面的``调试'', 然后点``开始执行(不调试)''.

``\texttt{Hello, world!}''被打出来啦!

右击右边``解决方案资源管理器''中的``Source Files'', 选择``添加''$\rightarrow$``新建项\dots'', 可以新建文件, 默认在\texttt{HelloWorld}文件夹里的\texttt{HelloWorld}文件夹里.右击任意一个文件的文件名后, 可以点击``删除''或``重命名''进行删除或重命名操作.

关掉Visual Studio, 重新打开, 左边多出了个HelloWorld.sln, 点它就能回到编辑界面了.

\section[GNU Fortran Compiler]{GNU Fortran Compiler (Gfortran)}

\subsection{安装}

Windows系统安装方式如下.\footnote{
    以下是直接安装MinGW-w64来安装Gfortran的, 但如果有Python, 也许能直接通过Pip~(或Conda)来安装Gfortran~(或MinGW-w64, MSYS2, \dots)? 我也不知\dots
}
\begin{enumerate}
    \item 访问~\href{https://github.com/niXman/mingw-builds-binaries/releases}
    {MinGW-Builds-binaries releases}, 选择最新的release进入.
    \item 选带posix的; 64位系统选带x86\_64的, 32位系统选带i686的; Win 10及以上选带ucrt的, Win 10以下选带msvcrt的. 下载并解压.
    \item 把解压出来的名为\texttt{mingw64}的文件夹剪切到随便哪个目录. 暂称粘贴到的目录为\texttt{[dir]}.
    \item 在系统环境变量\texttt{Path}中加入\texttt{[dir]\bs{}mingw64\bs{}bin}. 举个例子, 如果刚才粘贴到\texttt{C:\bs{}Program Files}, 就加入\texttt{C:\bs{}Program Files\bs{}mingw64\bs{}bin}.
    \item 下载~\href{https://code.visualstudio.com/sha/download?build=stable&os=win32-x64-user}
    {Visual Studio Code}~并安装.
    \item 打开Visual Studio Code, 点击左边四个正方形飞出一个的图标, 搜索C/C++, Modern Fortran和Code Runner并安装.
    \item[] 如果已经装了Python, 装Modern Fortran前先用pip\footnote{有conda当然用conda啦$\!\text{\~{}}$}装fortls.
    \item 按Ctrl+Shift+P, 然后选择``Preferences: Open Settings (JSON)'', 打开名为\texttt{settings.json}的JSON文件.
    \item 在\texttt{settings.json}里加入下面这些键值对.
    \begin{lstlisting}
"code-runner.executorMap": {
    "FortranFreeForm":
    "[cmd]",
    "fortran_fixed-form":
    "[cmd]"
},
"code-runner.runInTerminal": true,
"code-runner.saveAllFilesBeforeRun": true
    \end{lstlisting}
\end{enumerate}

最后一步其实和在VS Code的设置里点点点是一样滴, 我其实就是在设置里凭感觉点点点后把\texttt{settings.json}里的东东弄出来了, 如果玩代码的时候感觉不舒服, 自己解决, 我可不管. 注意
\begin{itemize}
    \item \texttt{[cmd]}要替换成编译并执行程序的整套命令.可以参考\ref{use_gfortran}小节, 按自己的需求替换, 也可以戳\href{https://zhuanlan.zhihu.com/p/362328064}{这里}搬运现成的.
    \item \texttt{[dir]}要替换成上文提到的\texttt{[dir]}.
\end{itemize}

还有加入键值对的时候要符合JSON的规则. 举两个例子.
如果一开始\texttt{settings.json}里是空的, 加入键值对后可能长这样.
\begin{lstlisting}
{
  "code-runner.executorMap": {
    "FortranFreeForm":
    "cd $dir; gfortran *.f*; if($?){.\\a.exe}",
    "fortran_fixed-form":
    "cd $dir; gfortran *.f*; if($?){.\\a.exe}"
  },
  "code-runner.runInTerminal": true,
  "code-runner.saveAllFilesBeforeRun": true
}
\end{lstlisting}
如果一开始\texttt{settings.json}长这样,
\begin{lstlisting}
{
  "editor.wordWrap": "wordWrapColumn",
  "editor.wordWrapColumn": 80,
  "workbench.colorTheme": "Red"
}
\end{lstlisting}
加入键值对后就可能长这样.
\begin{lstlisting}
{
  "editor.wordWrap": "wordWrapColumn",
  "editor.wordWrapColumn": 80,
  "code-runner.executorMap": {
    "FortranFreeForm":
    "cd $dir; gfortran *.f*; if($?){.\\a.exe}",
    "fortran_fixed-form":
    "cd $dir; gfortran *.f*; if($?){.\\a.exe}"
  },
  "code-runner.runInTerminal": true,
  "code-runner.saveAllFilesBeforeRun": true,
  "workbench.colorTheme": "Red"
}
\end{lstlisting}
注意倒二行最后多了一个逗号. 键值对间要用逗号隔开.

\subsection{使用}\label{use_gfortran}

\subsubsection{使用CLI}
Gfortran的用法和\ref{use_ifort}小节中Ifort的用法非常像, 不同点是:
\begin{itemize}
  \item 使用\texttt{gfortran}命令代替\texttt{ifort}命令;
  \item 中间过程生成的是\texttt{.o}文件,不是\texttt{.obj}文件;
  \item 默认生成文件\texttt{a.exe},且不会生成\texttt{.o}文件.
\end{itemize}

直接打开命令行(cmd或powershell), 即可使用\texttt{gfortran}命令.

在VS Code中, 可以直接点上面``Terminal''后点``New Terminal''来调出命令行. 默认调出的是powershell. 若想让默认调出的命令行是cmd, 可在\texttt{settings.json}里加入下面这个键值对.
\begin{lstlisting}
"terminal.integrated.shell.windows":
"C:\\Windows\\System32\\cmd.exe"
\end{lstlisting}

\subsubsection{使用GUI}
点击右上方的白色三角儿即可编译并执行! 除非之前\texttt{[cmd]}设错啦$\!\text{\~{}}$

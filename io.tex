\chapter{输入与输出}\label{io}

之前我们都只是小打小闹而已, 现在我们来干一些大活儿. 我们来玩玩\href{https://cdsarc.cds.unistra.fr/viz-bin/cat/I/355}{Gaia DR3}. 我们想知道Gaia DR3里的哪个源距银道面最远且最远有多远, 比较不动脑筋地做这件事呢, 我们想知道哪个源的$\left\lvert \sin b\right\rvert/p$最大, 其中$p$是源的视差, $b$是源的银纬. 但这里有一个问题, 就是Gaia DR3显然太大了, 同志们得交个几十杯奶茶钱的网费, 所以我们先用Gaia DR3的一个小样本试试水.

首先我们要在\href{https://cdsarc.cds.unistra.fr/viz-bin/cat/I/355}{这个页面}里找到\href{https://cdsarc.cds.unistra.fr/viz-bin/ReadMe/I/355?format=html&tex=true}{ReadMe}点进去, 找到``File Summary''.
\begin{lstlisting}
   File Summary:
   --------------------------------------------------------
    FileName      Lrecl  Records   Explanations
   --------------------------------------------------------
   ReadMe            80        .   This file
   gaiadr3.sam     1788     1000   Gaia DR3 source catalog
                                   (1811709771 sources)
                            ......
\end{lstlisting}
从这个``File Summary''可知, \texttt{gaiadr3.sam}是星表\footnote{认不得英文词儿的请上\href{https://nadc.china-vo.org/astrodict/}{天文学名词词典}查询.}, 一共1000行, 每行1788个字符. 不对呀, 后面明明写着一共1811709771个源嘛, 怎么只有1000行? 那当然是因为这只是个1000个源的样本啦, 文件名后缀可是\texttt{.sam}呢. 于是乎我们要在\href{https://cdsarc.cds.unistra.fr/viz-bin/cat/I/355}{这个页面}里找到\href{https://cdsarc.cds.unistra.fr/ftp/I/355}{FTP}, 里头找到\href{https://cdsarc.cds.unistra.fr/ftp/I/355/gaiadr3.sam.gz}{\texttt{gaiadr3.sam.gz}}下载下来然后解压得到\texttt{gaiadr3.sam}.

回到ReadMe, 下面有\texttt{gaiadr3.sam}的``Byte-by-byte Description''.
\begin{lstlisting}

   Byte-by-byte Description of file: gaiadr3.sam
   -----------...-------...--------------------------------
      Bytes   ... Units ... Explanations
   -----------...-------...--------------------------------
       ...         ...         ......
       1-  28 ... ---   ... Unique source designation ...
       ...         ...         ......
     129- 137 ... mas   ... ? Parallax (parallax)
       ...         ...         ......
     987-1001 ... deg   ... Galactic latitude (b)
       ...         ...         ......
\end{lstlisting}
从这个``Byte-by-byte Description''可知, \texttt{gaiadr3.sam}的每一行, 第1--28个字符都是源的编号, 第129--137个字符都是源的视差(单位为毫角秒), 第987--1001个字符都是源的银纬(单位为度). 好, 现在就造轮子!\label{gaiadr3.sam}
\begin{lstlisting}
program main
    use iso_fortran_env, only: dp => real64
    implicit none
    integer,parameter :: sam_len=1000
    real(dp),parameter :: mas2rad=acos(-1.0_dp)/(180*3600000)
    real(dp),parameter :: deg2rad=acos(-1.0_dp)/(180)
    real(dp) :: p(sam_len), b(sam_len)
    real(dp) :: d(sam_len)
    p = p*mas2rad
    b = b*deg2rad
    d = abs(sin(b))/p ! Unit: AU
    print *, maxloc(d), maxval(d)
end program main
\end{lstlisting}
这轮子现在有个大问题, 我们需要把星表里的数据转移到数组\texttt{p}和\texttt{b}中, 难道用列选择将星表里的数据复制粘贴到轮子里? 好家伙, 一个轮子两千多行, 其中两千行是数据, 要是处理原始星表, 就有三十六亿行是数据, 这不是要崩溃\footnote{不仅我们要崩溃, 文本编辑器也要崩溃\dots}? 这时候我们就需要让Fortran自己干转移数据的事情.

\section{文件}

我们平常所说的文件, Fortran称其为外部文件(external file), 不过一般来说外部文件得是纯文本文件\footnote{某些编译器可能有器规来应对二进制文件. 不知什么是纯文本文件和二进制文件的同志们赶紧恶补.}. 既然是纯文本文件, 就相当于一个字符串, 说到字符串, 就让我们想到字符串变量. Fortran称字符串变量为内部文件(internal file), 将外部文件与内部文件合称为文件(file).

我们需要让Fortran知道我们要用什么文件. 内部文件Fortran是认得的, 因为是自家的字符串变量嘛, 但外部文件Fortran不认得. 为了让Fortran认得外部文件, 我们要将外部文件打开(open). 下面这个轮子, Fortran会在第三行后认得\texttt{test.txt}, 这个\texttt{test.txt}应该和编译完最后生成的\texttt{.exe}文件\footnote{注意不是源代码文件.}在一个目录里.
\begin{lstlisting}
program main
    implicit none
    open(10, file='test.txt')
end program main
\end{lstlisting}
其中\texttt{10}这个位置填的是文件单位(file unit), 说白了就是一个编号, 这编号必须是非负整数, 而且最好不小于\texttt{10}, 因为小于\texttt{10}的编号可能有些奇奇怪怪的用途. \texttt{file=}后加的是文件路径, 不知什么是文件路径的同志们请自行补习. 上个轮子的第三行就是让Fortran认得文件路径为\texttt{test.txt}的文件, 并且称其为10号文件.

使用完外部文件后我们应当将其关闭(close), 也就是让Fortran忘记外部文件. 我们在上面那个轮子中加上一行, Fortran就会在第四行后忘记10号文件(即\texttt{test.txt}).
\begin{lstlisting}
program main
    implicit none
    open(10, file='test.txt')
    close(10)
end program main
\end{lstlisting}

每当我们暂时不需要使用外部文件时, 我们都应该将其关闭, 事实证明这是一个好习惯.

\begin{convention}
    在不需要使用外部文件的时候将其关闭.
\end{convention}

注意, 内部文件是不需要也没法打开或关闭的.

上面的轮子和下面的轮子等价.
\begin{lstlisting}
program main
    implicit none
    open(10, file='test.txt', status='UNKNOWN')
    close(10)
end program main
\end{lstlisting}
其中\texttt{status='UNKNOWN'}的意思是这个轮子在打开文件的时候进行了一波操作, 但这波操作是编译器自己决定的, 也就是说不同的编译器可能会有不同的操作, 所以不写\texttt{status=...}有点危险\footnote{如果摸透了编译器会怎么决定的话, 偷个懒不写也罢了.\label{no_add}}. 我们可以把\texttt{'UNKNOWN'}换成\texttt{'OLD'}, \texttt{'NEW'}, 或\texttt{'REPLACE'}. 换成\texttt{'OLD'}的话, \texttt{test.txt}必须在轮子运行前存在, 运行时直接打开. 换成\texttt{'NEW'}的话, \texttt{test.txt}必须在轮子运行前不存在, 运行时, Fortran会自己造一个空白\texttt{test.txt}文件并打开. 换成\texttt{'REPLACE'}的话, 如果\texttt{test.txt}在轮子运行前不存在, 则还是造一个空白\texttt{test.txt}文件并打开, 如果\texttt{test.txt}在轮子运行前存在, 则会把原来的\texttt{test.txt}直接删掉, 再造一个新的空白\texttt{test.txt}文件并打开. 请同志们自行造轮子实验上述操作.

\section{读取与写入}

读取(read)是把文件里的内容变成Fortran数据实体, 写入(write)是把Fortran数据实体变成文件里的内容. 让我们来看下面这个轮子.
\begin{lstlisting}
program main
    implicit none
    real :: e, pi
    open(10, file='test.txt', status='REPLACE', &
         action='WRITE', position='APPEND')
    write(10,*) exp(1.0), acos(-1.0)
    close(10)


    open(10, file='test.txt', status='OLD', &
         action='READ', position='REWIND')
    read(10,*) e, pi
    print *, e, pi
    close(10)
end program main
\end{lstlisting}
其中\texttt{write(10,*) ...}那行是把\texttt{exp(1.0)}和\texttt{acos(-1.0)}写入到10号文件里去, 所以打开\texttt{test.txt}就会看到$e$和$\pi$. \texttt{read(10,*) ...}那行则是把10号文件里的内容读取出来赋值给\texttt{e}和\texttt{pi}, 所以最后会输出$e$和$\pi$.

上面这个轮子还有些细节. \texttt{action=}后的字符串如果是\texttt{'READ'}, 表示打开的文件是只读的, 不能写入. \texttt{action=}后的字符串如果是\texttt{'WRITE'}, 表示打开的文件是只写的, 不能读取. \texttt{action=}后的字符串如果是\texttt{'READWRITE'}, 表示打开的文件是读写的, 读取写入皆可. 这和之前\ref{arguments}小节的只读参量只写参量读写参量是很像的. 如果不加\texttt{action=...}, 则是编译器自己决定是只读的只写的还是读写的, 这又有点危险了, 所以\texttt{action=...}还是要加的\footnote{同脚注\ref{no_add}.}.

\texttt{position=...}则指明打开文件后的``文件定位'', 加\texttt{position=...}的原因还是不加的话文件定位会由编译器自己决定\footnote{同脚注\ref{no_add}.}. 同志们不需知道文件定位是什么, 只需记得只写文件加\texttt{position='APPEND'}, 写入时会写入在文件的最后, 只读文件加\texttt{position='REWIND'}, 读取时会从文件的开头读取. 读写文件嘛, 我们选择不玩儿. 下面用一个长长的轮子来详细讲解.
\begin{lstlisting}
program main
    implicit none
    real :: e, pi
    real :: val
    e = exp(1.0)
    pi = acos(-1.0)
    open(10, file='test.txt', status='REPLACE', &
         action='WRITE', position='APPEND')
    close(10)


    open(10, file='test.txt', status='OLD', &
         action='WRITE', position='APPEND')
    write(10,*) e+pi
    write(10,*) e-pi
    close(10)
    open(10, file='test.txt', status='OLD', &
         action='WRITE', position='APPEND')
    write(10,*) e*pi
    write(10,*) e/pi
    close(10)
    open(10, file='test.txt', status='OLD', &
         action='READ', position='REWIND')
    read(10,*) val
    print *, val
    read(10,*) val
    print *, val
    close(10)
    open(10, file='test.txt', status='OLD', &
         action='READ', position='REWIND')
    read(10,*) val
    print *, val
    read(10,*) val
    print *, val
    close(10)
end program main
\end{lstlisting}
第一次打开关闭文件, 因为\texttt{status='REPLACE'}, 后面也没有写入, 所以最后得到一个空白文件. 第二次打开关闭文件, 先写入$e+\pi$, $e+\pi$后面再写入$e-\pi$. 第三次打开关闭文件, 在文件最后先写入$e\cdot\pi$, 再写入$e/\pi$, 所以最后文件里依次是$e+\pi$, $e-\pi$, $e\cdot\pi$, $e/\pi$. 第四次打开关闭文件, 先读取文件开头的$e+\pi$赋值给\texttt{val}, 再读取下面的$e-\pi$赋值给\texttt{val}. 第五次打开关闭文件, 还是先读取文件开头的$e+\pi$赋值给\texttt{val}, 再读取下面的$e-\pi$赋值给\texttt{val}, 所以最后输出的依次是$e+\pi$, $e-\pi$, $e+\pi$, $e-\pi$.

注意每次\texttt{read}或\texttt{write}后, 下次\texttt{read}或\texttt{write}都会\uline{从下一行开始}. 下面这个轮子, 第一次\texttt{write}后\texttt{1}和\texttt{2}写入在第一行, 第二次\texttt{write}后\texttt{3}和\texttt{4}写入在第二行, 第一次\texttt{read}从第一行开始, 因为后面只跟着一个\texttt{val1}, 所以\texttt{val1}为\texttt{1}, 第二次\texttt{read}从第二行开始, 因为后面只跟着一个\texttt{val2}, 所以\texttt{val2}为\texttt{3}, \texttt{2}和\texttt{4}则被无视!
\begin{lstlisting}
program main
    implicit none
    integer :: val1, val2
    open(10, file='test.txt', status='REPLACE', &
         action='WRITE', position='APPEND')
    write(10,*) 1, 2
    write(10,*) 3, 4
    close(10)
    open(10, file='test.txt', status='OLD', &
         action='READ', position='REWIND')
    read(10,*) val1 ! val1=1, while 2 is ignored!
    read(10,*) val2 ! val2=3, while 4 is ignored!
    print *, val1, val2
    close(10)
end program main
\end{lstlisting}

如果读取或写入时碰上数组, 则会把数组里的所有元素按元素顺序挨个读取或写入, 比如下面这个轮子, 先挨个写入1到9, 再写入10, 读取则是反向操作. 特别提醒, \texttt{one2nine(1,3)}是\texttt{7}, \texttt{one2nine(3,1)}是\texttt{3}, 同志们要是迷惑了请自行复习第\ref{fortran_array}章数组元素顺序的内容.
\begin{lstlisting}
program main
    implicit none
    integer :: i
    integer :: one2nine(3,3), ten
    open(10, file='test.txt', status='REPLACE', &
         action='WRITE', position='APPEND')
    write(10,*) reshape([(i,i=1,9)],[3,3]), 10
    close(10)
    open(10, file='test.txt', status='OLD', &
         action='READ', position='REWIND')
    read(10,*) one2nine, ten
    print *, one2nine, ten
    print *, one2nine(1,3), one2nine(3,1)
    close(10)
end program main
\end{lstlisting}

把文件单位换成字符串变量即可读取和写入内部文件了. 注意内部文件是不需要也没法打开或关闭的, 所以没法加\texttt{position=...}, 每次读取或写入都是从内部文件的开头第一个字符开始的, 不论之前读取写入几次. 下面这个轮子, Ifx可以把$e^\pi$写入\texttt{f}并读取至\texttt{val}\footnote{这就实现了数字型和字符型的相互转换.}, 但Gfortran不行, 要让Gfortran行, \texttt{f}的长度必须大于\texttt{16}. 原因见\ref{fortran_edit_descriptor}节.\label{internal_file}
\begin{lstlisting}
program main
    implicit none
    character(16) :: f
    real :: val
    write(f,*) exp(1.0)**acos(-1.0)
    print *, f
    read(f,*) val
    print *, val
end program main
\end{lstlisting}

我们还可以再把字符串变量换成\texttt{*}, 输出的话\texttt{*}通常代表电脑屏幕. 比如下面这个轮子就会把\texttt{Hello, world!}``写入电脑屏幕'', 电脑屏幕上就出现\texttt{Hello, world!}了. 要是问我\texttt{write(*,*)}和\texttt{print *,}有什么区别, 我的回答是没有区别\dots
\begin{lstlisting}
program main
    implicit none
    write(*,*) 'Hello, world!'
    print *, 'Bye bye, world!'
end program main
\end{lstlisting}
输入的话\texttt{*}则通常代表键盘. 比如下面这个轮子, 运行到第五行时程序会停下来, 我们在命令行(cmd呀powershell呀之类的东东)中可以打字儿, 假设打入\texttt{98 54e1}然后回车, \texttt{98}和\texttt{54e1}就赋值给\texttt{mat(1,1)}和\texttt{mat(1,2)}了, 第七行会干和第五行一样的事儿, 但语法更简单, 假设打入\texttt{7.6 3.2e0}然后回车, \texttt{7.6}和\texttt{3.2e0}就赋值给\texttt{mat(2,1)}和\texttt{mat(2,2)}了, 最后输出矩阵行列式. 注意``\texttt{*}''也是不需要且没法打开或关闭的.
\begin{lstlisting}
program main
    implicit none
    real :: mat(2,2)
    print *, 'Calculate determinant (det) of 2x2 matrix'
    print *, 'row 1 of matrix:'
    read(*,*) mat(1,1), mat(1,2)
    print *, 'row 2 of matrix:'
    read *, mat(2,:)
    print *, 'det:', mat(1,1)*mat(2,2)-mat(1,2)*mat(2,1)
end program main
\end{lstlisting}

相信同志们现在懂得如何读取和写入外部文件, 内部文件和``\texttt{*}''了, 不过同志们还需要注意一些细节问题. 在上个轮子中, 我一开始就打出了``算$2\times2$矩阵的行列式''的提示, 告诉用这个轮子的人这个轮子会干什么, 后面输入和输出是什么也有提示. 如果不这么干, 用轮子的人估计会一脸懵, 轮子在干什么, 自己要干什么, 最后得到的又是什么统统弄不懂, 哪怕是造轮子的人自己可能也会懵, 忘了自己干了什么在干什么该干什么\footnote{如果轮子只是自己用, 还敢赌自己能晓得轮子在干什么, 那想偷懒就偷懒呗.}.

\begin{convention}
    输出充足的与程序功能及输入输出有关的提示信息.
\end{convention}

\section{编辑符}\label{fortran_edit_descriptor}

如果我们分别用Ifx和Gfortran跑下面这个处心积虑造出来的简单轮子, 我们会发现结果的格式有点区别, Ifx的结果是用科学计数法表示的而Gfortran的不是.
\begin{lstlisting}

program main
    implicit none
    print *, 7.0e7
end program main
\end{lstlisting}
这是由于我们让编译器自己决定输入输出的格式, 有时这会出事情, 比如之前第\pageref{internal_file}页有个轮子, Ifx跑得了但Gfortran跑不了, 原因在于Gfortran在把\texttt{exp(1.0)**acos(-1.0)}写入内部文件\texttt{f}时会在数字前后补上些空格之类的, 按Gfortran自己的规定, 最少要写入17个字符(包含1个换行符), 而\texttt{f}长度只有16不够长, Ifx规定不同, 就没这个问题.

我们可以自己规定输入输出的格式, 比如下面这个轮子\footnote{Ifx编译时会出现一个``remark'', 是些建议, 我们选择无视, 不过听从Ifx的建议也是好事.}, 第一个输出的结果一定不是科学计数法表示的, 第二个一定是. 
\begin{lstlisting}
program main
    implicit none
    print "(F10.1)", 7.0e7
    print "(ES6.1E1)", 7.0e7
end program main
\end{lstlisting}
再比如修改第\pageref{internal_file}页那个轮子可得下面这个轮子, Gfortran也是可以跑的.
\begin{lstlisting}
program main
    implicit none
    character(16) :: f
    real :: val
    write(f,"(F15.12)") exp(1.0)**acos(-1.0)
    print *, f
    read(f,"(F15.12)") val
    print *, val
end program main
\end{lstlisting}
可见自己规定格式就是把原来的\texttt{*}换成一个字符串, 这个字符串里是\texttt{(...)}, 称为格式声明(format specification), \texttt{*}则是代表编译器自己决定格式.

我们回归一开始玩Gaia DR3遇到的问题, 现在我们可以把第\pageref{gaiadr3.sam}页的轮子补成下面的样子, 其中倒数第二行我们写\texttt{format}后跟格式声明(注意格式声明只是\texttt{(...)}, 不带引号), 前面加标号, 这样我们就能在倒数第三行用标号代替字符串了\footnote{同志们这么写的时候, 带标号的行最好就在用标号的行的下面, 不然查格式时真的是会找不到的\dots}. 不过呢, 同志们会发现怎么输出的最大距离是无穷大, 那是因为星表里有些源根本没有视差的数据, 硬读出来是$0$, 看来研究还是没那么容易做的\dots
\begin{lstlisting}
program main
    use iso_fortran_env, only: dp => real64
    implicit none
    integer,parameter :: sam_len=1000
    real(dp),parameter :: mas2rad=acos(-1.0_dp)/(180*3600000)
    real(dp),parameter :: deg2rad=acos(-1.0_dp)/(180)
    real(dp) :: p(sam_len), b(sam_len)
    real(dp) :: d(sam_len)
    integer :: i
    open(10, file='gaiadr3.sam', status='OLD', &
         action='READ', position='REWIND')
    do i = 1, sam_len
        read(10,"(128X,F9.4,849X,F15.11)") p(i), b(i)
    end do
    close(10)
    p = p*mas2rad
    b = b*deg2rad
    d = abs(sin(b))/p
    print 1000, maxloc(d), maxval(d)
    1000 format ('Object: ',I3,', ',ES11.4,' AU')
end program main
\end{lstlisting}

每个格式声明都由\texttt{()}里的一堆编辑符(edit descriptor)组成, 编辑符间用\texttt{,}隔开, 每个编辑符都代指一个操作, 比如上面的轮子读取时编辑符依次是\texttt{128X}, \texttt{F9.4}, \texttt{849X}, \texttt{F15.11}, 依次代表跳过128个字符不读, 读占9个字符的4位小数的实数, 跳过849个字符不读, 读占15个字符的11位小数的实数. 话说我是怎么知道这么读就行的? 我们再次打开\href{https://cdsarc.cds.unistra.fr/viz-bin/ReadMe/I/355?format=html&tex=true}{ReadMe}里的\texttt{gaiadr3.sam}的``Byte-by-byte Description'', 里头写着.
\begin{lstlisting}
  Byte-by-byte Description of file: gaiadr3.sam
  ------------------------...-------------------------------
     Bytes   Format Units ... Explanations
  ------------------------...-------------------------------
      ...     ....   ...         ......
      1-  28  A28   ---   ... Unique source designation ...
      ...     ....   ...         ......
    129- 137  F9.4  mas   ... ? Parallax (parallax)
      ...     ....   ...         ......
    987-1001 F15.11 deg   ... Galactic latitude (b)
      ...     ....   ...         ......
\end{lstlisting}
同志们看表里分明写着视差的格式是``F9.4'', 银纬的格式是``F15.11'', 这不就是编辑符么. 首先视差从第129个字符开始, 所以我们要先跳过前128个字符, 写上\texttt{128X}, 然后把\texttt{F9.4}无脑抄过去, 然后本来是读到第138个字符, 但银纬从第987个字符开始, 所以要再跳过$987-138=849$个字符, 写上\texttt{849X}, 然后无脑抄上\texttt{F15.11}, 因为每次\texttt{read}完再\texttt{read}都会从下一行开始, 所以后面的字符就不用管了. 给同志们留一个看完本章后的作业: 再次修改上面的轮子, 输出第\texttt{maxloc(d)}个源的``Unique source designation''.

如果我们把一些编辑符用\texttt{()}起来, 前面加个数, 数是几就表示把\texttt{()}里的编辑符重复几遍, 比如下面这个轮子写入和读取时格式声明是一样的(要不然就不知道读出什么东西了).
\begin{lstlisting}

program main
    implicit none
    real :: e, pi
    real :: a, s, m, d
    e = exp(1.0)
    pi = acos(-1.0)
    open(10, file='test.txt', status='REPLACE', &
         action='WRITE', position='APPEND')
    write(10,1006) e+pi, e-pi, e*pi, e/pi
    1006 format (F6.4,ES11.4,F6.4,ES11.4)
    close(10)
    open(10, file='test.txt', status='OLD', &
         action='READ', position='REWIND')
    read(10,1005) a, s, m, d
    1005 format (2(F6.4,ES11.4))
    print *, a, s, m, d
    close(10)
end program main
\end{lstlisting}
\texttt{()}还可以嵌套, 比如下面这个轮子两次写入时格式声明是一样的.
\begin{lstlisting}
program main
    implicit none
    ! A staff.
    open(10, file='test.txt', status='REPLACE', &
         action='WRITE', position='APPEND')
    write(10,1001) 'OXOX', 'OXXO', 'OX-X', 'OXXX'
    1001 format ('X','O','X','O','}',A4,'}', &
                 'X','O','X','O','}',A4,'}', &
                 'X','O','X','O','}',A4,'}', &
                 'X','O','X','O','}',A4,'}')
    close(10)
    open(10, file='test.txt', status='OLD', &
         action='WRITE', position='APPEND')
    write(10,1002) 'OXOX', 'OXXO', 'OX-X', 'OXXX'
    1002 format (4(2('X','O'),'}',A4,'}'))
    close(10)
end program main
\end{lstlisting}

编辑符又分三大类: 数据编辑符(data edit descriptor), 控制编辑符(control edit descriptor), 字符串编辑符(character string edit descriptor). 接下来我们一个个扒.

\subsection{数据编辑符}

数据编辑符和读取与写入时的数据实体是一一对应的, 比如下面这个轮子, \texttt{A4}对应\texttt{'ello'}, \texttt{A5}对应\texttt{'world'}, 其他编辑符不是数据编辑符, 没有对应的数据实体.
\begin{lstlisting}
program main
    implicit none
    open(10, file='test.txt', status='REPLACE', &
         action='WRITE', position='APPEND')
    write(10,"('H',A4,',',1X,A5,'!')") 'ello', 'world'
    close(10)
end program main
\end{lstlisting}
每个数据编辑符还都对应于一种数据类型, 比如下面这个轮子照道理是跑不得的, 因为\texttt{I1}是整型编辑符而\texttt{0.0}是实型的, 但Ifx居然能跑, 可恶的器规又出现了\dots
\begin{lstlisting}
program main
    implicit none
    print "(I1)", 0.0
end program main
\end{lstlisting}

数据编辑符都可以直接在前面加数来表示重复, 比如下面这个轮子三个格式声明全都是一样的.
\begin{lstlisting}
program main
    implicit none
    ! Violin Strings.
    open(10, file='test.txt', status='REPLACE', &
         action='WRITE', position='APPEND')
    write(10,"(A,A,A,A)") 'G', 'D', 'A', 'E'
    write(10,"(4(A))") 'G', 'D', 'A', 'E'
    write(10,"(4A)") 'G', 'D', 'A', 'E'
    close(10)
end program main
\end{lstlisting}

\subsubsection{整型编辑符}

I$w$编辑符表示一共输出$w$个字符. 我们需要保证$w>0$. 下面这个轮子, 前面几次写入是正常的, 但最后一次写入的是一堆\texttt{*}, 因为\texttt{1000000}分明是个7位数, 却只能输出6个字符, 编译器只能摆烂了.
\begin{lstlisting}
program main
    implicit none
    open(10, file='test.txt', status='REPLACE', &
         action='WRITE', position='APPEND')
    write(10,"(I6)") 1000
    write(10,"(I6)") 10000
    write(10,"(I6)") 100000
    write(10,"(I6)") 1000000
    close(10)
end program main
\end{lstlisting}
下面这个轮子, 后两次写入编译器都摆烂了, 因为``\texttt{-}''也占1个字符, 千万千万要注意!
\begin{lstlisting}
program main
    implicit none
    open(10, file='test.txt', status='REPLACE', &
         action='WRITE', position='APPEND')

    write(10,"(I6)") -1000
    write(10,"(I6)") -10000
    write(10,"(I6)") -100000
    write(10,"(I6)") -1000000
    close(10)
end program main
\end{lstlisting}

I$w.m$编辑符则表示一共输出$w$个字符, 其中数字字符(\texttt{0}--\texttt{9})至少$m$个, 当然必须$m\leqslant  w$. 下面这个轮子, \texttt{1000}只占4个字符, 所以前面要补上一个\texttt{0}, \texttt{10000}占5个字符, 正常输出, \texttt{100000}占6个字符, 也正常输出, 因为是\uline{至少}输出5个字符, \texttt{1000000}还是一堆\texttt{*}.
\begin{lstlisting}
program main
    implicit none
    open(10, file='test.txt', status='REPLACE', &
         action='WRITE', position='APPEND')

    write(10,"(I6.5)") 1000
    write(10,"(I6.5)") 10000
    write(10,"(I6.5)") 100000
    write(10,"(I6.5)") 1000000
    close(10)
end program main
\end{lstlisting}
特别注意输入时I$w.m$的$.m$会被无视, 也就是说I$w.m$等价于I$w$. 下面这个轮子是能正常运作的, 因为读取的时候I$6.5$等价于I$6$.
\begin{lstlisting}
program main
    implicit none
    integer :: k, dak, hk
    open(10, file='test.txt', status='REPLACE', &
         action='WRITE', position='APPEND')
    write(10,"(I6)") 1000
    write(10,"(I6)") 10000
    write(10,"(I6)") 100000
    close(10)

    open(10, file='test.txt', status='OLD', &
        action='READ', position='REWIND')
    read(10,"(I6.5)") k
    read(10,"(I6.5)") dak
    read(10,"(I6.5)") hk
    print *, k, dak, hk
    close(10)
end program main
\end{lstlisting}
反过来, 下面这个轮子也是能正常运作的, 虽然写入文件的时候\texttt{1000}前加了\texttt{0}, 但读取的时候开头的\texttt{0}都会被无视.
\begin{lstlisting}
program main
    implicit none
    integer :: k, dak, hk

    open(10, file='test.txt', status='REPLACE', &
         action='WRITE', position='APPEND')
    write(10,"(I6.5)") 1000
    write(10,"(I6.5)") 10000
    write(10,"(I6.5)") 100000
    close(10)
    open(10, file='test.txt', status='OLD', &
        action='READ', position='REWIND')
    read(10,"(I6)") k
    read(10,"(I6)") dak
    read(10,"(I6)") hk
    print *, k, dak, hk
    close(10)
end program main
\end{lstlisting}
但下面这个轮子就不对了, 因为写入6个字符但只读取5个字符, 这意味着最后的\texttt{0}没被读取, 结果\texttt{1000}, \texttt{10000}, \texttt{100000}被读成\texttt{100}, \texttt{1000}, \texttt{10000}.
\begin{lstlisting}
program main
    implicit none
    integer :: k, dak, hk
    open(10, file='test.txt', status='REPLACE', &
         action='WRITE', position='APPEND')
    write(10,"(I6.5)") 1000
    write(10,"(I6.5)") 10000
    write(10,"(I6.5)") 100000
    close(10)
    open(10, file='test.txt', status='OLD', &
        action='READ', position='REWIND')
    read(10,"(I5)") k
    read(10,"(I5)") dak
    read(10,"(I5)") hk
    print *, k, dak, hk ! Wrong!
    close(10)
end program main
\end{lstlisting}

\subsubsection{实型编辑符}

F$w.d$编辑符表示一共输出$w$个字符, 其中小数部分$d$个字符, 我们需要保证$w>d\geqslant0$. 下面这个轮子, 第一个输出是正常的, 第二个要输出``\texttt{-12.}''再加2个字符, 一共6个字符, 但却只能输出5个, 编译器又摆烂了.
\begin{lstlisting}
program main
    implicit none
    print "(F5.2)", 12.3456789
    print "(F5.2)", -12.3456789
end program main
\end{lstlisting}

输入的时候F$w.d$的$.d$则会被无视, 但$.d\,$\uline{不能被省略}. 下面这个轮子, F$11.99$看着很鬼, 一共11个字符, 小数部分99个? 但$.99$会被无视, 反正就是读$11$个字符然后赋值给\texttt{val}完事, 所以轮子是跑得的. 在下面的轮子中我们还写入双精度后读取成单精度, 这也没问题, 读取和写入可以跨种别.
\begin{lstlisting}
program main
    use iso_fortran_env, only: dp => real64
    implicit none
    real :: val
    open(10, file='test.txt', status='REPLACE', &
         action='WRITE', position='APPEND')
    write(10,"(F11.9)") 0.123456789_dp
    close(10)
    open(10, file='test.txt', status='OLD', &
        action='READ', position='REWIND')
    read(10,"(F11.99)") val
    print *, val
    close(10)
end program main
\end{lstlisting}
但下面这个轮子的结果是不对的, 因为写入和读取的\texttt{1000}没有小数点, 在没有小数点的时候, $.d$复活了, 编译器会认为读取到的字符的最后$d$个是小数部分, 读到\texttt{1000}, 最右边\texttt{0}是小数部分, 前面\texttt{100}是小数部分, 当然错啦!
\begin{lstlisting}

program main
    implicit none
    real :: val
    open(10, file='test.txt', status='REPLACE', &
         action='WRITE', position='APPEND')
    write(10,"(I4)") 1000
    close(10)
    open(10, file='test.txt', status='OLD', &
        action='READ', position='REWIND')
    read(10,"(F4.1)") val
    print *, val ! Wrong!
    close(10)
end program main
\end{lstlisting}

有时我们会碰到一些奇奇怪怪的数: $+\infty$, $-\infty$, 和$\text{NaN}$. $+\infty$和$-\infty$好理解, $\text{NaN}$名为``非数'', 表示``Not a Number'', 遇到什么不定式呀多值函数呀结果连$\pm\infty$都没法表示的时候结果就是$\text{NaN}$. Fortran规定输入输出时用前面加正负号的字符串\texttt{INF}或\texttt{INFINITY}表示$\pm\infty$, 用字符串\texttt{NAN}表示$\text{NaN}$, 这些字符串不分大小写. 如果遇到$\pm\infty$或$\text{NaN}$, 则不论输入输出F$w.d$的$.d$都会被无视\footnote{虽然Fortran官方规则中没写明, 但想来是这样的.\label{edit_IEEE}}, 比如下面的轮子里$d$可以随便乱写, 只需保证$d\geqslant0$.
\begin{lstlisting}
program main
    implicit none
    real :: one, zero
    real :: val1, val2, val3, val4
    one = 1.0
    zero = 0.0
    open(10, file='test.txt', status='REPLACE', &
         action='WRITE', position='APPEND')
    write (10,"(F9.12)") +(one/zero)
    write (10,"(F9.34)") -(one/zero)
    write (10,"(F9.56)") +(zero/zero)
    write (10,"(F9.78)") -(zero/zero)
    close(10)
    open(10, file='test.txt', status='OLD', &
        action='READ', position='REWIND')
    read(10,"(F9.87)") val1
    read(10,"(F9.65)") val2
    read(10,"(F9.43)") val3
    read(10,"(F9.21)") val4
    print *, val1, val2, val3, val4
    close(10)
end program main
\end{lstlisting}

F编辑符经常会不大好用, 举个例子, 假如我们要输出$1.234\times10^{-3}$, $1.234$, $1.234\times10^{3}$量级不同的三个数, 编辑符都用F$5.3$, 那就只有$1.234$的输出是比较正常的, $1.234\times10^{-3}$有效数字丢了, $1.234\times10^{3}$干脆输出不了, 但麻烦的是实际的观测数据量级有差别是经常出现的事情.
\begin{lstlisting}
program main
    implicit none
    print "(F5.3)", 1.234e-3 ! output: 0.001
    print "(F5.3)", 1.234    ! output: 1.234
    print "(F5.3)", 1.234e+3 ! output: *****
end program main
\end{lstlisting}
这时我们可以用E编辑符, E$w.d$表示一共输出$w$个字符, 输出时先将实数表示成$a\times10^{n}$, $0.1\leqslant a<1$, $n$为整数, 把$a$转换成小数部分占$d$个字符的字符串\texttt{a}, $n$转换成开头带正负号的字符串\texttt{n}, 然后输出字符串\texttt{a//'E'//n}. 我们需要保证$w>d\geqslant0$. 最后输出的字符串\texttt{a//'E'//n}中, 开头的\texttt{0}和中间的\texttt{E}可省略, 但输出的字符串去掉\texttt{a}后剩下的部分必占4个字符, 所以保证$w\geqslant 3+d+4$一般就没什么事情了. 假如我们想保留4位有效数字, 编辑符设成E$11.4$就好啦.
\begin{lstlisting}
program main
    implicit none
    print "(E11.4)", 1.234e-3 ! output:  0.1234E-02
    print "(E11.4)", 1.234    ! output:  0.1234E+01
    print "(E11.4)", 1.234e+3 ! output:  0.1234E+04
end program main
\end{lstlisting}

用E$w.d$编辑符的时候, \texttt{n}是三位数则\texttt{E}必须省略, \texttt{n}是四位数就只能罢工了, 虽然平常我们基本上不会用上这么极端的数\dots
\begin{lstlisting}
program main
    use iso_fortran_env, only: qp => real128
    implicit none
    print "(E11.4)", 1e10_qp   ! output:  0.1000E+11
    print "(E11.4)", 1e100_qp  ! output:  0.1000+101
    print "(E11.4)", 1e1000_qp ! output: ***********
end program main
\end{lstlisting}
这时我们可以用E$w.d$E$e$编辑符, E$e$表示\texttt{n}为正负号后接$e$个数字的字符串, 其他和E$w.d$相同, 除了\texttt{E}不可省略外. 这时我们需要额外保证$e>0$, 再保证$w\geqslant 3+d+2+e$一般就没什么事情了.
\begin{lstlisting}
program main
    use iso_fortran_env, only: qp => real128
    implicit none
    print "(E13.4E4)", 1e10_qp   ! output:  0.1000E+0011
    print "(E13.4E4)", 1e100_qp  ! output:  0.1000E+0101
    print "(E13.4E4)", 1e1000_qp ! output:  0.1000E+1001
end program main
\end{lstlisting}
不过E$w.d$E0也是合法的编辑符, 其中E0表示$e$等于$n$的位数, 但俺手里的Gfortran不认这个编辑符, 它out了!
\begin{lstlisting}
program main
    use iso_fortran_env, only: qp => real128
    implicit none
    print "(E13.4E0)", 1e10_qp   ! output:    0.1000E+11
    print "(E13.4E0)", 1e100_qp  ! output:   0.1000E+101
    print "(E13.4E0)", 1e1000_qp ! output:  0.1000E+1001
end program main
\end{lstlisting}

和F编辑符类似, 输入的时候$.d$, E$e$, E0都会被无视, $.d\,$不能被省略. 不仅如此, 用F编辑符写入后还能用E编辑符读取, 用E编辑符写入后也能用F编辑符读取, 读取的时候字符\texttt{E}还不分大小写.
\begin{lstlisting}
program main
    implicit none
    character(13+8+1) :: f
    real :: val_fe, val_ef
    integer :: i
    write(f,"(F13.1,E8.1)") 1e10, 1e10
    do i = 1, len(f)
        if (f(i:i)=='E') then
            f(i:i) = 'e'
        end if
    end do
    print *, f
    read(f,"(E13.13,F8.8)") val_fe, val_ef
    print *, val_fe, val_ef
end program main
\end{lstlisting}

E编辑符也可以应付$\pm\infty$和$\text{NaN}$, 遇到$\pm\infty$或$\text{NaN}$的时候$d$和$e$被无视, 其他和F编辑符相同\footnote{同脚注\ref{edit_IEEE}}, 也就是说我们又可以乱写了.
\begin{lstlisting}
program main
    implicit none
    real :: one, zero
    real :: val1, val2, val3, val4
    one = 1.0
    zero = 0.0
    open(10, file='test.txt', status='REPLACE', &
         action='WRITE', position='APPEND')
    write (10,"(F9.12)") +(one/zero)
    write (10,"(E9.34)") -(one/zero)
    write (10,"(E9.5E6)") +(zero/zero)
    write (10,"(E9.78E0)") -(zero/zero)
    close(10)
    open(10, file='test.txt', status='OLD', &
        action='READ', position='REWIND')
    read(10,"(E9.87E0)") val1
    read(10,"(E9.6E5)") val2
    read(10,"(E9.43)") val3
    read(10,"(F9.21)") val4
    print *, val1, val2, val3, val4
    close(10)
end program main
\end{lstlisting}

E编辑符蛮好用, 就是最后输出的结果不太符合俺们的习惯, 因为不是用科学计数法表示的. 我们只要把E换成ES, 结果就是科学计数法表示的了, 也就是说$1\leqslant a<10$. 我们还可以把E换成EN, 这样的结果是用工程计数法表示的, $1\leqslant a<1000$且$n$能被$3$整除, 这样单位换算就会比较方便. 除了$a$和$n$不同外E编辑符, ES编辑符, EN编辑符没有区别.
\begin{lstlisting}
program main
    implicit none
    print "(E7.1E1)", 10000.0
    print "(ES7.1E1)", 10000.0
    print "(EN7.1E1)", 10000.0
end program main
\end{lstlisting}

\subsubsection{复型编辑符}

复型编辑符是没有的, 输出复数的时候, 永远是把实部和虚部分别输出, 我们可以分别给实部和虚部加实型编辑符.
\begin{lstlisting}
program main
    implicit none
    print "(F4.1,E8.1)", (0.1,1.0)
end program main
\end{lstlisting}
输入也是一样的道理, 注意输入的时候实型编辑符是可以乱来的.
\begin{lstlisting}
program main
    implicit none
    complex :: z

    open(10, file='test.txt', status='REPLACE', &
         action='WRITE', position='APPEND')
    write(10,"(F4.1,E8.1)") (0.1,1.0)
    close(10)
    open(10, file='test.txt', status='OLD', &
        action='READ', position='REWIND')
    read(10,"(E4.1,F8.1)") z
    print *, z
    close(10)
end program main
\end{lstlisting}

\subsubsection{字符型编辑符}

A$w$编辑符表示一共输出$w$个字符. 设字符串长度为$l$, 如果$l>w$, 则只输出字符串最左边$w$个字符, 如果$l<w$, 则先输出$w-l$个空格再输出$w$个字符\footnote{这里是左补空格, 字符串赋值(\ref{fortran_assignment}节)是右补空格.}. A编辑符则表示$w=l$.
\begin{lstlisting}
program main
    implicit none
    open(10, file='test.txt', status='REPLACE', &
         action='WRITE', position='APPEND')
    write(10,"(A4)") 'hello'
    write(10,"(A5)") 'hello'
    write(10,"(A6)") 'hello'
    close(10)
    open(10, file='test.txt', status='OLD', &
         action='WRITE', position='APPEND')
    write(10,"(A)") 'hello'
    write(10,"(A)") 'hellohello'
    write(10,"(A)") 'hellohellohello'
    close(10)
end program main
\end{lstlisting}

输入的话情况比较复杂. 首先同志们要认定每一行最后都有无数个空格, 下面这个轮子, 写入完第一行是\texttt{1234567890}, 读取完\texttt{c}当然等于\texttt{'12345'}, 第二行是\texttt{1}, 后面没了, 同学们要认定\texttt{1}后面跟着无数个空格, 所以读取完\texttt{c}等于\texttt{'1    '}
\begin{lstlisting}
program main
    implicit none
    character(5) :: c
    open(10, file='test.txt', status='REPLACE', &
         action='WRITE', position='APPEND')
    write(10,"(A)") '1234567890'
    write(10,"(A)") '1'
    close(10)
    open(10, file='test.txt', status='OLD', &
        action='READ', position='REWIND')
    read(10,"(A5)") c
    print "(2A)", c, '}'
    read(10,"(A5)") c
    print "(2A)", c, '}'
    close(10)
end program main
\end{lstlisting}
然后A编辑符表示$w=l$这点不变.
\begin{lstlisting}
program main
    implicit none
    character(4) :: sc
    character(6) :: lc
    open(10, file='test.txt', status='REPLACE', &
         action='WRITE', position='APPEND')
    write(10,"(A)") '1234567890'
    write(10,"(A)") '1234567890'
    close(10)
    open(10, file='test.txt', status='OLD', &
        action='READ', position='REWIND')

    read(10,"(A)") sc ! (A4)
    print "(2A)", sc, '}'
    read(10,"(A)") lc ! (A6)
    print "(2A)", lc, '}'
    close(10)
end program main
\end{lstlisting}
A$w$编辑符则表示读取$w$个字符. 若$l<w$, 则赋值最右边$l$个字符\footnote{这里是赋值最右边的字符, 字符串赋值(\ref{fortran_assignment}节)是赋值最左边的字符.}, 若$l>w$, 则赋值$w$个字符后跟$l-w$个空格.
\begin{lstlisting}
program main
    implicit none
    character(4) :: sc
    character(5) :: nc
    character(6) :: lc
    open(10, file='test.txt', status='REPLACE', &
         action='WRITE', position='APPEND')
    write(10,"(A)") '1234567890'
    write(10,"(A)") '1234567890'
    write(10,"(A)") '1234567890'
    close(10)
    open(10, file='test.txt', status='OLD', &
        action='READ', position='REWIND')
    read(10,"(A5)") sc
    print "(2A)", sc, '}'
    read(10,"(A5)") nc
    print "(2A)", nc, '}'
    read(10,"(A5)") lc
    print "(2A)", lc, '}'
    close(10)
end program main
\end{lstlisting}

可见字符型编辑符十分让人头大, 如果老师敢考我们就当即暴动!

注意字符型编辑符和\ref{character_string_edit_descriptor}节的字符串编辑符是八竿子打不着的.

\subsubsection{逻辑型编辑符}

L$w$表示一共输出$w$个字符, 前面$w-1$个是空格, 最后1个是\texttt{T}或\texttt{F}, \texttt{T}代表\texttt{.true.}, \texttt{F}代表\texttt{.false.}.
\begin{lstlisting}
program main
    implicit none
    print "(L7)", .true.
    print "(L7)", .false.
end program main
\end{lstlisting}
输入的时候, 字符\texttt{T}和\texttt{F}, 字符串\texttt{TRUE}和\texttt{FALSE}, 字符串\texttt{.TRUE.}和\texttt{.FALSE.}都代表\texttt{.true.}和\texttt{.false.}, 而且不分大小写.
\begin{lstlisting}
program main
    implicit none
    logical :: true, false
    open(10, file='test.txt', status='REPLACE', &
         action='WRITE', position='APPEND')
    write(10,"(A)") 't f '
    write(10,"(A)") 'true false'
    write(10,"(A)") '.true. .false.'
    close(10)
    open(10, file='test.txt', status='OLD', &
        action='READ', position='REWIND')
    read(10,"(2L2)") true, false
    print "(2L2)", true, false
    read(10,"(2L5)") true, false
    print "(2L5)", true, false
    read(10,"(2L7)") true, false
    print "(2L7)", true, false
    close(10)
end program main
\end{lstlisting}

\subsection{控制编辑符}

$n$X编辑符表示把``文件定位''右移$n$位, 这通常等价于输出$n$个空格, 但如果$n$X编辑符后没有数据编辑符或字符串编辑符, 则$n$X编辑符相当于没有.
\begin{lstlisting}
program main
    implicit none
    open(10, file='test.txt', status='REPLACE', &
         action='WRITE', position='APPEND')
    write(10,"(2(A,1X))") 'X', 'descriptor'
    close(10)
end program main
\end{lstlisting}
输入的时候$n$X编辑符则表示跳过$n$个字符不读.
\begin{lstlisting}
program main
    implicit none
    character(5) :: world
    open(10, file='test.txt', status='REPLACE', &
         action='WRITE', position='APPEND')
    write(10,"(A)") 'Hello, world!'
    close(10)
    open(10, file='test.txt', status='OLD', &
        action='READ', position='REWIND')
    read(10,"(7X,A5)") world
    print "(A)", world
    close(10)
end program main
\end{lstlisting}

/编辑符表示接下来从下一行第一个字符开始读取或写入.
\begin{lstlisting}
program main
    implicit none
    character(5) :: hello, world
    open(10, file='test.txt', status='REPLACE', &
         action='WRITE', position='APPEND')
    write(10,"(A,/,A)") 'hello', 'world'
    close(10)
    open(10, file='test.txt', status='OLD', &
        action='READ', position='REWIND')
    read(10,"(A,/,A)") hello, world
    print "(A,/,A)", hello, world
    close(10)
end program main
\end{lstlisting}

SS编辑符表示之后输出正数时最开头都不带正号, SP编辑符则表示都带正号.\footnote{``S''代表``sign''. ``S''代表``suppress'', ``P''代表``plus''.}
\begin{lstlisting}
program main
    implicit none
    print 1001, 0.1, 2.3, 4.5, 6.7, 8.9
    1001 format (SS,F5.1,F5.1,SP,F5.1,SS,F5.1,F5.1)
    print 1002, 0.1, 2.3, 4.5, 6.7, 8.9
    1002 format (SP,F5.1,F5.1,SS,F5.1,SP,F5.1,F5.1)
end program main
\end{lstlisting}
注意正号会占一个字符, 编译器有可能因此罢工.
\begin{lstlisting}
program main
    implicit none
    print "(SS,F3.1)", 1.0 ! 1.0
    print "(SP,F3.1)", 1.0 ! ***
end program main
\end{lstlisting}
输入的时候SS编辑符和SP编辑符不起作用, 我们又可以乱来了.
\begin{lstlisting}
program main
    implicit none
    real :: val0, val1, val2, val3, val4
    open(10, file='test.txt', status='REPLACE', &
         action='WRITE', position='APPEND')

    write(10,1061) 0.1, 2.3, 4.5, 6.7, 8.9
    1061 format (SS,2F5.1,SP,F5.1,SS,2F5.1)
    write(10,1062) 0.1, 2.3, 4.5, 6.7, 8.9
    1062 format (SP,2F5.1,SS,F5.1,SP,2F5.1)
    close(10)
    open(10, file='test.txt', status='OLD', &
         action='READ', position='REWIND')
    read(10,1051) val0, val1, val2, val3, val4
    1051 format (SP,F5.1,SS,3F5.1,SP,F5.1)
    print *, val0, val1, val2, val3, val4
    read(10,1052) val0, val1, val2, val3, val4
    1052 format (SS,F5.1,SP,3F5.1,SS,F5.1)
    print *, val0, val1, val2, val3, val4
    close(10)
end program main
\end{lstlisting}

在输出实数的时候, 我们只能输出若干位有效数字, 所以这里有个舍入的问题. 比如测量值我们希望四舍五入但不确定度我们希望向上舍入. 我们可以用RU, RD, RZ, RC编辑符, 这四个编辑符分别表示之后输出实数时都向上舍入, 都向下舍入, 都向零舍入, 都四舍五入.\footnote{``R''代表``round''. ``U''代表``up'', ``D''代表``down'', ``Z''代表``zero'', ``C''代表``compatible''.}
\begin{lstlisting}
program main
    implicit none
    print "(RU,4F5.1,/,RD,4F5.1,/,RZ,4F5.1,/,RC,4F5.1)", &
        -5.6789, -0.1234, +0.1234, +5.6789, &
        -5.6789, -0.1234, +0.1234, +5.6789, &
        -5.6789, -0.1234, +0.1234, +5.6789, &
        -5.6789, -0.1234, +0.1234, +5.6789
end program main
\end{lstlisting}
输入时RU, RD, RZ, RC编辑符也起作用, 因为用字符串表示的实数都是形如$\sum_{i=m_\text{min}}^{m_\text{max}}10^{i}$的实数, 而电脑是二进制的, 读取后实数都必须是形如$\sum_{i=n_\text{min}}^{n_\text{max}}2^{i}$的实数, 所以也得舍入.

\subsection{字符串编辑符}\label{character_string_edit_descriptor}

字符串编辑符就是一个字符串, 作用就是输出这个字符串.
\begin{lstlisting}
program main
    implicit none
    print "(A,', world!')", 'Hello'
    print "('Hello, ',A,'!')", 'world'
end program main
\end{lstlisting}
读取和写入的时候, 我们其实可以不加任何数据实体, 所以可以秀波操作.
\begin{lstlisting}
program main
    implicit none
    print "('Hello, world!')",
end program main
\end{lstlisting}
输入的时候不能用字符串编辑符, 我们可以用$n$X编辑符代替.
\begin{lstlisting}
program main
    implicit none
    character :: the_end
    open(10, file='test.txt', status='REPLACE', &
         action='WRITE', position='APPEND')
    write(10,"('Hello, world',A1)") '!'
    close(10)
    open(10, file='test.txt', status='OLD', &
         action='READ', position='REWIND')
    read(10,"(12X,A1)") the_end
    print *, the_end
    close(10)
end program main
\end{lstlisting}

\chapter{过程}\label{fortran_procedure}

假如我们需要用Fortran算阶乘$10!$, 那还是很容易滴.
\begin{verbatim}
program main
    implicit none
    integer :: i, p
    p = 1
    do i = 1, 10
        p = p*i
    end do
    print *, p
end program main
\end{verbatim}

假如我们需要用Fortran算组合数$\text{C}_7^3=\frac{7!}{3!(7-3)!}$, 那就有点麻烦.
\begin{verbatim}
program main
    implicit none
    integer :: i, c1, c2, c3, c
    c1 = 1
    do i = 1, 7
        c1 = c1*i
    end do
    c2 = 1
    do i = 1, 3
        c2 = c2*i
    end do


    c3 = 1
    do i = 1, 7-3
        c3 = c3*i
    end do
    c = c1/(c2*c3)
    print *, c
end program main
\end{verbatim}
麻烦的地方在于那个阶乘老是要\verb|do|来\verb|do|去, 不过就\verb|do|三回, 还能活.

假如我们需要用Fortran算CG系数$\left\langle 3,2;5,4|7,6\right\rangle $,
\begin{align*}
    \left\langle j_1,m_1;j_2,m_2|j_3,m_3\right\rangle&=\delta_{m_3,m_1+m_2}\Big[(2j_3+1)\\
    &\cdot\frac{(j_1+j_2-j_3)!(j_2+j_3-j_1)!(j_3+j_1-j_2)!}{(j_1+j_2+j_3+1)!}
    \\
    &\cdot\prod_{i=1,2,3}(j_i+m_i)!(j_i-m_i)!\Big]^{1/2}\sum_{\nu\in F}[(-1)^{\nu}\nu!\\
    &\cdot(j_1+j_2-j_3-\nu)!\\
    &\cdot(j_1-m_1-\nu)!(j_2+m_2-\nu)!\\
    &\cdot(j_3-j_1-m_2+\nu)!(j_3-j_2+m_1+\nu)!],
\end{align*}
那不知要\verb|do|多少回, 算个大头鬼哟! 不算了, 准备卸Fortran了!

桥豆麻袋, Fortran是有法子能偷懒滴(如果没有我第一个卸Fortran), 比如算$\text{C}_7^3$可以这样.
\begin{verbatim}
program main
    implicit none
    integer :: c, factorial
    c = factorial(7)/(factorial(3)*factorial(7-3))
    print *, c
end program main

function factorial(n) result(p)
    integer,intent(in) :: n
    integer :: p
    integer :: i 
    p = 1
    do i = 1, n
        p = p*i
    end do
end function factorial
\end{verbatim}
写成这样, 确实能少打几个字儿. 不知道我写的是什么也可以先猜猜猜看. 把\verb|function ...|到\verb|end function ...|单看成一个程序, \verb|result(p)|里的\verb|p|就是\verb|factorial(n)|里的\verb|n|的阶乘, 如是这般, \verb|factorial(7)|就是\verb|7|的阶乘, \verb|factorial(3)|就是\verb|3|的阶乘, \verb|factorial(7-3)|就是\verb|7-3|的阶乘, 非常完美! 那算CG系数也简单多了, 只要算\verb|X|的阶乘的时候无脑写上\verb|factorial(X)|就成了, 不用\verb|do|来\verb|do|去了!

于是乎我们便发现, 想要玩Fortran而不是被Fortran玩, 就必须懂过程(procedure). 过程的定义是``封装可以在程序执行期间直接调用的任意操作序列的实体'', 玄玄乎乎的, 我们不理它. 我们可以直接把过程理解成程序运行时的一个操作, 比如上面的例子中\verb|function ...|到\verb|end function ...|就是``计算\verb|n|的阶乘''这一操作. 使用过程后, 我们就进入面向过程程序设计(procedure-oriented programming, POP)阶段了. 啥叫面向过程呢? 众所周知, 置象于冰箱中, 步骤有三: 一开冰箱, 二塞大象, 三关冰箱. 用Fortran来写便这般.
\begin{verbatim}
program main
    ...
    implicit none
    ...
    call open_door()
    call put_in(elephant)
    call close_door()
end program main

subroutine open_door()
    ...
end subroutine open_door


subroutine put_in(what_put_in)
    ...
end subroutine put_in

subroutine close_door()
    ...
end subroutine close_door
\end{verbatim}
这样一看就知道这程序干仨事儿: 第一步\verb|call open_door()|开冰箱, 第二步\verb|call put_in(elephant)|塞大象, 第三步\verb|call close_door()|关冰箱. 至于到底咋么开的冰箱, 咋么塞的大象, 咋么关的冰箱, 看对应的\verb|subroutine|到\\\verb|end subroutine|
里咋写的就知道了. 把整个程序(冰箱塞大象)拆成一个个步骤(开冰箱, 塞大象, 关冰箱), 然后每个步骤造个过程, 这就是面向过程. 那为什么塞大象要写成\verb|put_in(elephant)|而不是\verb|put_elephant_in()|? 这是因为说不准以后要把别的东西放进冰箱, 真有那天, 把\verb|elephant|换掉就行了, 哦, 还要把大象拿出来, 再多造个过程\dots

\section{外部过程}

我们先细掰一些基本概念. 任何过程都是要用一堆字符表示的, 这堆字符便称为子程序(subprogram). 过程和子程序的关系, 就像程序和源代码(见\ref{fortran_program}节), 法律和法律条文的关系一样, 后者是表述前者的一堆字符\footnote{今后不再区分过程和子程序.}. 子程序按摆的位置, 分为外部子程序(external subprogram), 内部子程序(internal subprogram)和模块子程序(module subprogram). 内部子程序瞅着没啥用还容易让同志们脑壳疼, 俺不打算讲, 模块子程序见第\ref{fortran_module}章, 本章只讲外部子程序. 子程序按长的样子, 又分为子例行子程序(subroutine subprogram)和函数子程序(function subprogram).

\subsection{子例行子程序}

我们来详细分析下面这个程序.
\begin{verbatim}


program main
    use iso_fortran_env, only: dp => real64
    implicit none
    real(dp) :: a, b, c
    real(dp) :: x, y, z
    a = 1.0_dp
    b = 2.0_dp
    c = 3.0_dp
    x = 4.0_dp
    y = 5.0_dp
    z = 6.0_dp
    call ab2bc_then_sumabc(x, y, z)
    print *, a, b, c
    print *, x, y, z
end program main

subroutine ab2bc_then_sumabc(a, b, c)
    use iso_fortran_env, only: dp => real64
    implicit none
    real(dp),intent(in) :: a
    real(dp),intent(inout) :: b
    real(dp),intent(out) :: c
    real(dp) :: s
    c = b
    b = a
    s = a+b+c
    print *, s
end subroutine ab2bc_then_sumabc
\end{verbatim}
\begin{enumerate}
    \item 从\verb|subroutine ...|到\verb|end subroutine ...|便是子例行子程序了. 这部分可以和主程序放在一个文件里, 顺序也随便, 但通常是单独放在另一个文件里.
    \item \verb|subroutine|和\verb|end subroutine|后的\verb|XX| (示例中为\verb|ab2bc_then_sumabc|)称为子例行子程序名. 一般存这个子程序的文件就会取成\verb|XX.f90| (比如示例中的子程序就可以存入名为\verb|ab2bc_then_sumabc.f90|的文件中). 当然, 子程序名得尽量表明子程序的功能.
    \item 子程序和主程序一样都是程序单元(见\ref{run_fortran}节), 一样得变量声明一波, 所以\verb|implicit none|得加, 要用种别的话\verb|use iso_fortran_env|也得加.
    \item 子程序里有三个变量\verb|a|, \verb|b|, \verb|c|, 我在声明时加了\verb|,intent(...)|, 这三个变量在\verb|ab2bc_then_sumabc|后的\verb|()|里出现, 在\verb|()|里出现的称为哑参量(dummy argument), 只有哑参量能在声明时加\verb|,intent(...)|. 我称加\verb|,intent(in)|的为只读(read-only)参量, 加\verb|,intent(inout)|的为读写(read-write)参量, 加\verb|,intent(out)|的为只写(write-only)参量.
    \item 主程序里也有三个变量\verb|a|, \verb|b|, \verb|c|, 但只是变量名和子程序里的\verb|a|, \verb|b|, \verb|c|一样, 实际上是不同的三个变量. 看我这示例, 子程序里\verb|a|, \verb|b|, \verb|c|变来变去, 花里胡哨, 主程序里\verb|a|, \verb|b|, \verb|c|岿然不动.
    \item 主程序里出现\verb|call ab2bc_then_sumabc(x, y, z)|, \verb|()|里的\verb|x|, \verb|y|, \verb|z|称为实参量(actual argument). 实参量可以是任意数据实体, 也就是说实参量可以是变量, 也可以是常量或其他东东. 子程序里三个哑参量排排坐, 主程序里三个实参量排排坐, 位置一样的哑参量和实参量(\verb|a|和\verb|x|, \verb|b|和\verb|y|, \verb|c|和\verb|z|)称为对应的. 对应的参量在程序运行时会相互赋值来赋值去, 这称为参量结合(argument association), 我们一般称为哑实结合.
    \item 现在分析示例程序的运行过程. 程序当然从\verb|program main|开始运行了. 按顺序一行一行运行, 前面不需讲解. 到\verb|call ...|, 就要说道说道了. 我们可以把主程序和子程序当成两个小人儿.\begin{enumerate}
        \item \verb|call ab2bc_then_sumabc(x, y, z)|: 跳到子程序的开头, 也就是\verb|subroutine ab2bc_then_sumabc(a, b, c)|. \verb|call ...|这里程序运行的操作称为``主程序调用(invoke/call)子程序''.
        \item \verb|subroutine ab2bc_then_sumabc(a, b, c)|: 子程序先按后面的变量声明语句声明好变量, 然后把所有只读的和读写的实参量赋值给对应的哑参量(主程序里的\verb|x|赋给子程序里的\verb|a|, 主程序里的\verb|y|赋给子程序里的\verb|b|, 这当然就是传说中的哑实结合啦).
        \item 向下一行一行运行, 直到\verb|end subroutine ab2bc_then_sumabc|. 啰嗦一下具体过程. 首先主程序里的\verb|x|赋给子程序里的\verb|a|, 主程序里的\verb|y|赋给子程序里的\verb|b|, 所以子程序里的\verb|a|为\verb|4.0_dp|, 子程序里的\verb|b|为\verb|5.0_dp|. 然后\verb|c = b|, 子程序里的\verb|c|为\verb|5.0_dp|, 然后\verb|b = a|, 子程序里的\verb|b|为\verb|4.0_dp|, 然后\verb|s = a+b+c|, \verb|s|为\verb|13.0_dp|, 最后输出\verb|s|的值\verb|13.0_dp|.
        \item \verb|end subroutine ab2bc_then_sumabc|: 把所有读写的和只写的哑参量赋值给对应的实参量(子程序里的\verb|b|赋给主程序里的\verb|y|, 子程序里的\verb|c|赋给主程序里的\verb|z|, 这当然也是传说中的哑实结合啦), 然后跳到\verb|call ab2bc_then_sumabc(x, y, z)|的下一行.
    \end{enumerate}然后主程序继续按顺序一行一行运行至\verb|end program main|结束. 再啰嗦啰嗦, 子程序里的\verb|b|赋给主程序里的\verb|y|, 子程序里的\verb|c|赋给主程序里的\verb|z|, 所以\verb|x|还是\verb|4.0_dp|, \verb|y|则变为\verb|4.0_dp|, \verb|z|则变为\verb|5.0_dp|.
\end{enumerate}

啊! 上面那个程序终于分析完毕! 不仅是主程序能调用子程序, 任何程序单元都能调用子程序, 所以我们还可以玩点更花的. 假如我们现在要算组合数$\text{C}_7^3$, 我们可以造一个主程序, 一个算组合数的子程序, 一个算阶乘的子程序, 然后让主程序调用算组合数的子程序, 算组合数的子程序调用算阶乘的子程序, 就像下面这样. 请同志们自己分析其运行过程.\label{fact_comb}
\begin{verbatim}
program main
    implicit none
    integer :: result
    call combinatorial(7, 3, result)
    print *, result
end program main

subroutine combinatorial(n, m, comb)
    implicit none
    integer,intent(in) :: n
    integer,intent(in) :: m
    integer,intent(out) :: comb
    integer :: a, b, c
    call factorial(n, a)
    call factorial(m, b)
    call factorial(n-m, c)
    comb = a/(b*c)
end subroutine combinatorial

subroutine factorial(n, fact)
    implicit none
    integer,intent(in) :: n
    integer,intent(out) :: fact
    integer :: i
    fact = 1
    do i = 1, n
        fact = fact*i
    end do
end subroutine factorial
\end{verbatim}

子程序灰常管用, 但也是要遵守一些禁令的. 首先哑实结合时, 哑参量和实参量的类型和种别\uline{都要相等}, 也就是说莫得类型种别转化了. 比如下面这个程序, Gfortran日常严格, 会出警告, 输出\verb|0.00000000|, Ifort日常宽松, 不出警告, 但也输出\verb|0.0000000E+00|\dots
\begin{verbatim}
program main
    use iso_fortran_env, only: sp => real32
    implicit none
    real(sp) :: a
    a = 10.0_sp
    call add_one(a)
    print *, a
end program main

subroutine add_one(a)
    use iso_fortran_env, only: dp => real64
    implicit none
    real(dp),intent(inout) :: a
    a = a+1.0_dp
end subroutine add_one
\end{verbatim}
然后只读的(加\verb|,intent(in)|的)哑参量不能在子程序运行的时候被赋值(哑实结合当然还是可以的), 比如下面这个程序跑不得, Ifort和Gfortran都是如此. 这个规则是非常适当的, 因为如果我们可以确定一个哑参量在程序中是不应当被赋值的, 我们就可以让其成为只读的, 这样如果我们一不小心写错了, 在程序中给这个哑参量赋值了, 就能马上查出来.
\begin{verbatim}
program main
    use iso_fortran_env, only: dp => real64
    implicit none
    real(dp) :: a
    a = 10.0_dp
    call add_one(a)
    print *, a
end program main

subroutine add_one(a)
    use iso_fortran_env, only: dp => real64
    implicit none
    real(dp),intent(in) :: a
    a = a+1.0_dp
end subroutine add_one
\end{verbatim}
注意间接的赋值也是不行的, 比如下面这个程序, \verb|call add_one_(a)|实际上给\verb|add_one|里的\verb|a|赋值了. 但这样``隐晦的''赋值, 编译器就不一定会查了, Gfortran会给警告, Ifort直接放行. 但无论如何这么写都是不对的!\label{secret_assignment}
\begin{verbatim}
program main
    use iso_fortran_env, only: dp => real64
    implicit none
    real(dp) :: a
    a = 10.0_dp
    call add_one(a)
    print *, a
end program main


subroutine add_one(a)
    use iso_fortran_env, only: dp => real64
    implicit none
    real(dp),intent(in) :: a
    call add_one_(a)
end subroutine add_one

subroutine add_one_(a)
    use iso_fortran_env, only: dp => real64
    implicit none
    real(dp),intent(inout) :: a
    a = a+1.0_dp
end subroutine add_one_
\end{verbatim}
还有只写的(加\verb|,intent(out)|的)哑参量, 在子程序运行的一开始是没有被赋值的, 所以下面这程序是天也不知会出什么结果的. 但Ifort和Gfortran有器规, 会把只写参量当成读写参量(我猜是这样了啦), 一开始也实参量赋值给哑参量了, 所以结果没事. 但无论如何这么写都是不对的!
\begin{verbatim}
program main
    use iso_fortran_env, only: dp => real64
    implicit none
    real(dp) :: a
    a = 10.0_dp
    call add_one(a)
    print *, a
end program main

subroutine add_one(a)
    ! Dummy argument may not be defined here!
    use iso_fortran_env, only: dp => real64
    implicit none
    real(dp),intent(out) :: a
    a = a+1.0_dp
end subroutine add_one
\end{verbatim}

正确标注哑参量为只读参量, 只写参量或读写参量\label{arguments}(加\verb|intent(...)|), 是灰常灰常必要的, 血泪教训告诉我们这么做能避免踩很多很多坑. 同志们一定不能怕麻烦, 老老实实一个个标注. 如果哑参量除哑实结合时外不被赋值, 就标只读, 如果哑参量在哑实结合时不需要被赋值, 就标只写, 剩下的标读写.

\begin{principle}
    正确标注每个哑参量为只读参量, 只写参量或读写参量.
\end{principle}

子程序还有很多禁令, 我安排同志们自己写程序测试, 编译器会告诉大家答案的. 比如, 子程序名能和主程序名一样嘛? 子程序名能和子程序里的变量的变量名一样嘛? 子程序名能和调用子程序的程序单元里的变量的变量名一样嘛? 子程序能直接或间接地调用自己嘛?\dots

\subsection{函数子程序}

如果同志们没被子例行子程序弄晕, 那理解函数子程序便轻而易举了. 我们还是写一个计算组合数的子例行子程序, 不过我用\verb|...|省略一部分, 想来同志们自己补上没问题.
\begin{verbatim}
program main
    implicit none
    integer :: result
    call combinatorial(7, 3, result)
    print *, result
end program main

subroutine combinatorial(n, m, comb)
    implicit none
    integer,intent(in) :: n
    integer,intent(in) :: m
    integer,intent(out) :: comb
    integer :: a, b, c
    integer :: i
    ...
    comb = a/(b*c)
end subroutine combinatorial
\end{verbatim}
上面这个程序完全可以用函数子程序改写成下面这样, 请同志们对比改写前后的样子.
\begin{verbatim}
program main
    implicit none
    integer :: combinatorial
    integer :: result
    result = combinatorial(7, 3)
    print *, result
end program main

function combinatorial(n, m) result(comb)
    implicit none
    integer,intent(in) :: n
    integer,intent(in) :: m
    integer :: comb
    integer :: a, b, c
    integer :: i
    ...
    comb = a/(b*c)
end function combinatorial
\end{verbatim}
\begin{enumerate}
    \item 从\verb|function ...|到\verb|end function ...|便是函数子程序了. 和子例行子程序一样, 可以和主程序放在一个文件里, 顺序也随便, 但通常是单独放在另一个文件里. \verb|function|和\verb|end function|后的\verb|XX| (示例中为\verb|combinatorial|)一样称为函数子程序名. 一般存这个子程序的文件也一样会取成\verb|XX.f90|.
    \item 函数子程序名后的\verb|()|里的当然是哑参量, \verb|()|后的\verb|result(...)|里的变量\verb|...| (示例中为\verb|comb|)相当于一个只写哑参量, 称为结果(result), 但结果本身不是哑参量. 声明结果时不需也不能加\verb|,intent(out)|.
    \item 调用函数子程序后, 结果赋值给函数名和其之后的括号及括号内的内容组成的整体. 示例中, 一开始主程序里\verb|7|和\verb|3|赋值给子程序里\verb|n|和\verb|m|, 最后子程序里结果\verb|comb|赋值给主程序里\verb|combinatorial(7, 3)|这一整个长串, \verb|combinatorial(7, 3)|就成为一个数据实体, 因此可以再赋值给主程序里的\verb|result|变量.
    \item 调用函数子程序前必须对函数子程序本身进行声明(声明的类型和种别是函数子程序的结果的类型和种别), 子例行子程序是不需要的. 比如示例程序的主程序加了\verb|integer :: combinatorial|一句, 因为整型是函数\verb|combinatorial|的结果\verb|comb|的类型.
\end{enumerate}
按照当今的规范, 我们必须保证函数子程序的所有哑参量都是只读的(结果不是哑参量). 如果不遵守此规范, 我保证同志们之后会无比头痛.

\begin{principle}
    标注函数子程序的所有哑参量为只读参量.
\end{principle}

使用函数子程序的好处是调用函数子程序后会生成一个数据实体, 经验表明多数情况下这样能让我们偷懒少打几个字, 即便使用函数子程序前必须多加一行函数子程序的声明. 我把之前第\pageref{fact_comb}页用计算阶乘的子程序计算组合数的程序用函数子程序改写如下, 同志们会不会觉得看着简单一点?
\begin{verbatim}
program main
    implicit none
    integer :: combinatorial
    print *, combinatorial(7, 3)
end program main

function combinatorial(n, m) result(comb)
    implicit none
    integer,intent(in) :: n
    integer,intent(in) :: m
    integer :: comb
    integer :: factorial
    comb = factorial(n)/(factorial(m)*factorial(n-m))
end function combinatorial

function factorial(n) result(fact)
    implicit none

    integer,intent(in) :: n
    integer :: fact
    integer :: i
    fact = 1
    do i = 1, n
        fact = fact*i
    end do
end function factorial
\end{verbatim}

\subsection{固有过程}

\newcommand{\ip}[1]{\href{https://fortranwiki.org/fortran/show/#1}{\texttt{#1}}}
为了让我们能快乐地玩轮子, 合格的Fortran编译器都已经自己造好了一大堆轮子, 称为固有过程(intrinsic procedure), 我们直接调用就可以了. 固有过程有哪些怎么用请猛戳\href{https://fortranwiki.org/fortran/show/Intrinsic+procedures}{这个链接}查询. 同志们造轮子前都应该先查查有没有已经造好的固有轮子可以用. 比如我们如果想算$\pi$, 如果我们很熟悉固有轮子的话, 我们就会想到有个轮子\verb|acos|, 是算反余弦的, 我们用它算$\arccos (-1)$即可. 注意, \verb|acos|是个函数, 要声明的.
\begin{verbatim}
program main
    use iso_fortran_env, only: qp => real128
    implicit none
    real(qp) :: acos
    print *, acos(-1.0_qp)
end program main
\end{verbatim}

\section{过程中的变量}

\subsection{parameter属性}\label{fortran_parameter}

我们在玩轮子的时候经常会遇到这么个问题, 比如我们需要反复进行一些和$\pi$有关的计算, 假设我们的程序以双精度运行, 我们会造一个实型双精度变量\verb|pi|, 然后赋值\verb|acos(-1.0_dp)|, 其中\verb|dp|为\verb|real64|, 这本来没什么问题, 但如果我们一不小心写错了程序, 悄咪咪地又给这个\verb|pi|赋了个别的值, 那就出了大问题, 而且编译器还不会告诉我们, 我们就会死都找不出哪里写错了. 如果我们把\verb|pi|改成\verb|3.141592653589793_dp|或\verb|acos(-1.0_dp)|, 那倒是不大会出错了, 但每次看轮子, 我们脑子都要反复判断出\verb|3.141592653589793_dp|\\或\verb|acos(-1.0_dp)|其实就是$\pi$, 我们的眼睛和脑子就会觉得好伤心好难过呜呜呜呜.

这时候我们就可以给\verb|pi|加上parameter属性\footnote{属性的含义见第\ref{fortran_attribute}章.}, 造一个具名常量(named constant). 比如下面这个程序, 我们在声明\verb|pi|时加了\verb|,parameter|, 并在声明的同时给\verb|pi|赋值\verb|acos(-1.0_dp)|, 这个\verb|pi|就永远是\verb|acos(-1.0_dp)|改不了了, 比如我们下面又把\verb|pi|赋成\verb|exp(1.0_dp)|, 合格的编译器就会立即跳出来把我们打回去重新改程序.
\begin{verbatim}
program main
    use iso_fortran_env, only: dp => real64
    implicit none
    real(dp),parameter :: pi=acos(-1.0_dp)
    pi = exp(1.0_dp) ! It is impossible to change pi!
end program main
\end{verbatim}
同志们可以自己尝试用子程序给具名常量赋值, 看看会出什么事, 可以把第\pageref{secret_assignment}页的``隐晦赋值''程序改一改来考验考验编译器.

\subsection{save属性}

和parameter属性有一丢丢像的是save属性. 下面这个程序, 连续输出三个\verb|1|, 这当然没有问题.
\begin{verbatim}
program main
    use iso_fortran_env, only: dp => real64
    implicit none
    call count()
    call count()
    call count()
end program main

subroutine count()
    implicit none
    integer :: n
    n = 0
    n = n+1
    print *, n
end subroutine count
\end{verbatim}
但下面这个程序输出的却是\verb|1|, \verb|2|, \verb|3|. 在这个程序里, 子程序\verb|count|里的\verb|n|声明的时候加了\verb|,save|, 并赋值\verb|0|. 第一次调用\verb|count|的时候, \verb|n|一开始是\verb|0|, 然后\verb|n = n+1|, \verb|n|就是\verb|1|. 而第二次调用\verb|count|的时候, \verb|n|一开始并没有重新被赋值成\verb|0|, 而是保存着上一次调用到最后的值\verb|1|, 所以再次\verb|n = n+1|后\verb|n|变成\verb|2|. 第三次调用\verb|count|的时候, \verb|n|一开始是\verb|2|, 所以最后是\verb|3|.
\begin{verbatim}
program main
    use iso_fortran_env, only: dp => real64
    implicit none
    call count() ! 0 -> 1
    call count() ! 1 -> 2
    call count() ! 2 -> 3
end program main

subroutine count()
    implicit none
    integer,save :: n = 0
    n = n+1
    print *, n
end subroutine count
\end{verbatim}
有的同志可能会尝试在变量声明的时候直接给变量赋值, 因为这样可以偷一点懒, 但这么做是非常危险的, 因为这么做的时候, 即使没加\verb|,save|, 变量也悄咪咪地带上save属性了. 比如下面这个程序和上面那个程序是一样的, 但因为没写\verb|,save|, 同志们可能就会忘记变量\verb|n|有save属性!
\begin{verbatim}
program main
    use iso_fortran_env, only: dp => real64
    implicit none


    call count()
    call count()
    call count()
end program main

subroutine count()
    implicit none
    integer :: n = 0 ! n has SAVE attribute, though no ,save is here!
    n = n+1
    print *, n
end subroutine count
\end{verbatim}
因此, 不要这么写, 除非前面已经标好\verb|,parameter|或\verb|,save|.

\begin{principle}
    除非已经明确标明变量有\emph{parameter}属性或\emph{save}属性, 否则不要在声明变量的时候直接给变量赋值.
\end{principle}

\subsection{过程中的数组}

在用子程序的时候碰上数组, 就会比较麻烦, 因为哑实结合还和数组的形状有关. 我们必须细掰细掰.

\subsubsection{显式形状数组}
一个$n$维向量$r=(r_1,\dots,r_n)$的$1$范数为$\left\lVert r\right\rVert_1:=\sum_{i=1}^n\left\lvert r_i\right\rvert $. 假如我们要算$(-1,2)$的$1$范数, 写个轮子还是很容易的.
\begin{verbatim}
program main
    use iso_fortran_env, only: dp => real64
    implicit none
    real(dp) :: r(2)
    real(dp) :: norm_1
    r = [-1.0_dp,2.0_dp]
    print *, norm_1(r)
end program main

function norm_1(r) result(norm)
    use iso_fortran_env, only: dp => real64
    implicit none
    real(dp),intent(in) :: r(2)
    real(dp) :: norm
    norm = sum(abs(r))
end function
\end{verbatim}
其中主程序里的\verb|r|和函数里的\verb|r|, 形状都大大咧咧的写在那里, 这就叫显式形状数组(explicit-shape array). 但我们上面这个程序大有问题, 如果我们要算$(-1,2,-3)$的$1$范数, 因为函数里的\verb|r|形状被定死为\verb|[2]|, 所以要出事儿. 这时我们可以这么写.\label{adjustable_array}
\begin{verbatim}
program main
    use iso_fortran_env, only: dp => real64
    implicit none
    real(dp) :: r_2(2), r_3(3)
    real(dp) :: norm_1
    r_2 = [-1.0_dp,2.0_dp]
    r_3 = [-1.0_dp,2.0_dp,-3.0_dp]
    print *, norm_1(r_2, size(r_2))
    print *, norm_1(r_3, size(r_3))
end program main

function norm_1(r, size_r) result(norm)
    use iso_fortran_env, only: dp => real64
    implicit none
    real(dp),intent(in) :: r(size_r)
    integer,intent(in) :: size_r
    real(dp) :: norm
    norm = sum(abs(r))
end function
\end{verbatim}
这个程序, 函数里哑参量\verb|r|的形状是由哑参量\verb|size_r|决定的, 这样的哑参量数组就叫可调数组(adjustable array), 定义为显式形状数组中的一种. 虽然\verb|size_r|的声明在\verb|r|下面, 但放心, 声明\verb|r|的时候会先查看\verb|size_r|的值的.

第\pageref{hw_2}页的小作业二是计算一个奇怪矩阵的所有元素的和, 我们可以把问题扩大点, 算任意$n\times n$的那种矩阵的所有元素的和, 我们则可以这么写.
\begin{verbatim}
program main
    use iso_fortran_env, only: dp => real64
    implicit none
    real(dp) :: strange_mat_element_sum
    print *, strange_mat_element_sum(50)
end program main

function strange_mat_element_sum(n) result(s)
    use iso_fortran_env, only: dp => real64
    implicit none
    integer,intent(in) :: n
    real(dp) :: s
    real(dp) :: mat(n,n)
    integer :: i, j
    do i = 1, n
        do j = 1, n
            mat(i,j) = sqrt(real(i+j-1))
        end do
    end do
    s = sum(mat)
end function
\end{verbatim}
这个程序, 函数里\verb|mat|的形状是由哑参量\verb|n|决定的, 这样的数组就叫自动数组(automatic array), 也定义为显式形状数组中的一种. 自动数组和可调数组的区别是可调数组一定是哑参量, 自动数组一定不是.

\subsubsection{假定形状数组}

之前第\pageref{adjustable_array}页用可调数组的轮子, 每次都要算数组的大小, 然后和子程序哑实结合, 还是麻烦. 用假定形状数组(assumed-shape array)就可以这么解决这个问题.
\begin{verbatim}

program main
    use iso_fortran_env, only: dp => real64
    implicit none
    real(dp) :: r_2(2), r_3(3)
    real(dp) :: norm_1
    r_2 = [-1.0_dp,2.0_dp]
    r_3 = [-1.0_dp,2.0_dp,-3.0_dp]
    print *, norm_1(r_2)
    print *, norm_1(r_3)
end program main

function norm_1(r) result(norm)
    use iso_fortran_env, only: dp => real64
    implicit none
    real(dp),intent(in) :: r(:)
    real(dp) :: norm
    norm = sum(abs(r))
end function
\end{verbatim}
在这个程序里, 函数中\verb|r|就是假定形状数组, 其声明的时候写\verb|r(:)|, 1个\verb|:|表示其必须是1维数组, 形状是其对应的实参量的形状, 这样就不用每次都计算实参量的大小然后哑实结合了. 不过上面这个轮子Gfortran是跑不了的, 因为按Fortran的语法, 有假定形状数组哑参量的过程, 必须带过程接口(见\ref{fortran_interface}节), 所以要写成下面这个样子才行. Ifort能跑, 但结果是错的\dots\label{assumed-shape_array_program}
\begin{verbatim}
program main
    use iso_fortran_env, only: dp => real64
    implicit none
    real(dp) :: r_2(2), r_3(3)
    interface
        function norm_1(r) result(norm)
            use iso_fortran_env, only: dp => real64
            implicit none


            real(dp),intent(in) :: r(:)
            real(dp) :: norm
        end function
    end interface
    r_2 = [-1.0_dp,2.0_dp]
    r_3 = [-1.0_dp,2.0_dp,-3.0_dp]
    print *, norm_1(r_2)
    print *, norm_1(r_3)
end program main

function norm_1(r) result(norm)
    use iso_fortran_env, only: dp => real64
    implicit none
    real(dp),intent(in) :: r(:)
    real(dp) :: norm
    norm = sum(abs(r))
end function
\end{verbatim}
使用假定形状数组的时候, 我们还可以指定假定形状数组每一维的下界. 比如下面这个程序, 函数里的\verb|r|是2维数组, 第1维下界是0, 第2维下界没写, 默认是1.
\begin{verbatim}
program main
    use iso_fortran_env, only: dp => real64
    implicit none
    real(dp) :: r_2(2,1), r_3(3,1)
    interface
        function norm_1(r) result(norm)
            use iso_fortran_env, only: dp => real64
            implicit none
            real(dp),intent(in) :: r(0:,:)
            real(dp) :: norm
        end function
    end interface

    r_2 = reshape([0.0_dp,-1.0_dp],[2,1])
    r_3 = reshape([0.0_dp,-1.0_dp,2.0_dp],[3,1])
    print *, norm_1(r_2)
    print *, norm_1(r_3)
end program main

function norm_1(r) result(norm)
    use iso_fortran_env, only: dp => real64
    implicit none
    real(dp),intent(in) :: r(0:,:)
    real(dp) :: norm
    norm = sum(abs(r))
end function
\end{verbatim}

\subsection{哑过程}
有的时候我们需要把子程序本身当成参量, 比如我们如果要造个定积分的轮子, 我们就要被积函数, 下界, 上界三个参量, 被积函数参量当然得是函数啦. 是哑参量的过程称为哑过程(dummy procedure). 下面就是个定积分轮子, 虽然看着非常复杂, 但确实能跑. 这个轮子将在\ref{fortran_interface}节中讲解.\label{dummy_procedure_program}
\begin{verbatim}
program main
    use iso_fortran_env, only: dp => real64
    implicit none
    interface
        function identity(x) result(y)
            use iso_fortran_env, only: dp => real64
            implicit none
            real(dp),intent(in) :: x
            real(dp) :: y
        end function
    end interface
    real(dp) :: integrate
    print *, integrate(identity, 0.0_dp, 1.0_dp)
end program main

function integrate(f, a, b) result(s) ! trapezoidal rule
    use iso_fortran_env, only: dp => real64
    implicit none
    abstract interface
        function func(x) result(y)
            use iso_fortran_env, only: dp => real64
            implicit none
            real(dp),intent(in) :: x
            real(dp) :: y
        end function
    end interface
    procedure(func) :: f
    real(dp),intent(in) :: a
    real(dp),intent(in) :: b
    real(dp) :: s
    real(dp) :: h
    integer :: i
    h = (b-a)/10000
    s = (f(a)+f(b))/2
    do i = 1, 9999
        s = s+f(a+i*h)
    end do
    s = s*h
end function integrate

function identity(x) result(y)
    use iso_fortran_env, only: dp => real64
    implicit none
    real(dp),intent(in) :: x
    real(dp) :: y
    y = x
end function identity
\end{verbatim}

\section{过程接口}\label{fortran_interface}

之前第\pageref{assumed-shape_array_program}页和第\pageref{dummy_procedure_program}页的轮子用了过程接口(procedure interface). 过程接口可分为三类: 特定接口(specific interface), 泛型接口(generic interface)和抽象接口(abstract interface).

\subsection{特定接口}

特定接口相当于过程的声明. 通常情况下, 子例行不需要声明, 函数也只需要声明结果就可以了. 但如果碰上了以下情形之一:
\begin{itemize}
    \item 过程有一个哑参量, 此哑参量满足下列条件之一:\begin{itemize}
        \item 是延迟长度字符型变量(见\ref{fortran_char}节)或延迟形状数组(见\ref{fortran_array_specification}节),
        \item 是假定形状数组,
    \end{itemize}
    \item 过程是函数且结果是数组,
    \item 过程是另一个过程的实参量,
\end{itemize}
那么就一定要加特定接口\footnote{一定要加特定接口的情形还有很多, 其他同志们暂时不需要掌握.}. 这不一定需要背, 编译器应该是要告诉我们的. 我们可以先不加接口, 然后如果编译器告诉我们``Explicit interface required for \dots''或``Expected a procedure for argument \dots''或其他七七八八的话, 那就是要加接口了.

特定接口都放在从\verb|interface|到\verb|end interface|的一整块里, 这一整块称为接口块(interface block), 一个接口块里可以放一堆接口. 接口是过程的一部分, 我们只要把过程的变量声明部分下面的执行部分都删掉, 然后变量声明部分, 除了哑参量和结果的声明, 其他声明都删掉, 然后复制粘贴到接口块里就可以了.

我们通过实战来熟悉这个过程. 假设我们现在想搞个算单位矩阵的轮子, 我们正常地这么一写, 结果跑不得.
\begin{verbatim}
program main
    use iso_fortran_env, only: dp => real64
    implicit none
    real(dp) :: i(3,3)
    i = eye(3)
end program main

function eye(n) result(mat)
    use iso_fortran_env, only: dp => real64
    implicit none
    integer,intent(in) :: n
    real(dp) :: mat(n,n)
    !----------------------------------------
    integer :: i
    mat = 0.0_dp
    do i = 1, n
        mat(i,i) = 1.0_dp
    end do
    !----------------------------------------
end function eye
\end{verbatim}
我们需要在主程序中加个接口. 我们先在主程序的声明部分写上\verb|interface|\\和\verb|end interface|, 然后把整个\verb|eye|子程序复制粘贴进去. (我们还可以用列选择\footnote{不知道什么是列选择的同志请自行了解.}来调整格式)
\begin{verbatim}
program main
    use iso_fortran_env, only: dp => real64
    implicit none
    interface
        function eye(n) result(mat)
            use iso_fortran_env, only: dp => real64
            implicit none
            integer,intent(in) :: n
            real(dp) :: mat(n,n)
            !----------------------------------------
            integer :: i
            mat = 0.0_dp

            do i = 1, n
                mat(i,i) = 1.0_dp
            end do
            !----------------------------------------
        end function eye
    end interface
    real(dp) :: i(3,3)
    i = eye(3)
end program main

function eye(n) result(mat)
    use iso_fortran_env, only: dp => real64
    implicit none
    integer,intent(in) :: n
    real(dp) :: mat(n,n)
    !----------------------------------------
    integer :: i
    mat = 0.0_dp
    do i = 1, n
        mat(i,i) = 1.0_dp
    end do
    !----------------------------------------
end function eye
\end{verbatim}
然后我们定睛一看, 两个注释行中间的部分, 第一行是\verb|i|的声明, \verb|i|既不是哑参量也不是结果, 所以删去, 后面几行是执行部分也删去. 删完以后就成下面这个样子.
\begin{verbatim}
program main
    use iso_fortran_env, only: dp => real64
    implicit none
    interface
        function eye(n) result(mat)
            use iso_fortran_env, only: dp => real64
            implicit none
            integer,intent(in) :: n
            real(dp) :: mat(n,n)
        end function eye
    end interface
    real(dp) :: i(3,3)
    i = eye(3)
end program main

function eye(n) result(mat)
    use iso_fortran_env, only: dp => real64
    implicit none
    integer,intent(in) :: n
    real(dp) :: mat(n,n)
    !----------------------------------------
    integer :: i
    mat = 0.0_dp
    do i = 1, n
        mat(i,i) = 1.0_dp
    end do
    !----------------------------------------
end function eye
\end{verbatim}
然后这个轮子就能跑了, 欧耶!

同志们会不会觉得这么做挺麻烦的? 俺也觉得, 可是这也是没办法的. Fortran在设置这个接口规则的时候可是经过深思熟虑的, 因为造编译器的凭他们的经验告诉我们, 有些情况(比如上面列出来的), 如果没有接口, 会出大事情. 不过还是有办法能让我们少费点脑子, 那就是使用模块(见第\ref{fortran_module}章), 但如果不需要加接口, 那么造一个模块反而比较费事\dots

这里还有个问题, 如果我们碰到个固有过程需要接口怎么办, 比如我们如果要算$\sin$的积分, 我们用固有过程\verb|sin|直接干算不得.
\begin{verbatim}
program main
    use iso_fortran_env, only: dp => real64
    implicit none
    real(dp) :: integrate

    print *, integrate(sin, 0.0_dp, 1.0_dp)
end program main

function integrate(f, a, b) result(s) ! trapezoidal rule
    use iso_fortran_env, only: dp => real64
    implicit none
    abstract interface
        function func(x) result(y)
            use iso_fortran_env, only: dp => real64
            implicit none
            real(dp),intent(in) :: x
            real(dp) :: y
        end function
    end interface
    procedure(func) :: f
    real(dp),intent(in) :: a
    real(dp),intent(in) :: b
    real(dp) :: s
    real(dp) :: h
    integer :: i
    h = (b-a)/10000
    s = (f(a)+f(b))/2
    do i = 1, 9999
        s = s+f(a+i*h)
    end do
    s = s*h
end function integrate
\end{verbatim}
最无脑的办法就是我们再造个过程, 把固有过程变成外部过程, 像下面这样, 然后再加上接口即可. 请同志们自己给下面这个程序补上接口.
\begin{verbatim}
program main
    use iso_fortran_env, only: dp => real64
    implicit none
    real(dp) :: integrate
    print *, integrate(sin_, 0.0_dp, 1.0_dp)
end program main

function integrate(f, a, b) result(s) ! trapezoidal rule
    use iso_fortran_env, only: dp => real64
    implicit none
    abstract interface
        function func(x) result(y)
            use iso_fortran_env, only: dp => real64
            implicit none
            real(dp),intent(in) :: x
            real(dp) :: y
        end function
    end interface
    procedure(func) :: f
    real(dp),intent(in) :: a
    real(dp),intent(in) :: b
    real(dp) :: s
    real(dp) :: h
    integer :: i
    h = (b-a)/10000
    s = (f(a)+f(b))/2
    do i = 1, 9999
        s = s+f(a+i*h)
    end do
    s = s*h
end function integrate

function sin_(x) result(y)
    use iso_fortran_env, only: dp => real64
    implicit none

    real(dp),intent(in) :: x
    real(dp) :: y
    y = sin(x)
end function sin_
\end{verbatim}

\subsection{泛型接口}

泛型接口是一个比较妙的东东. 假如我们现在要造个算符号函数$\sgn$的轮子, 那是轻而易举的.
\begin{verbatim}
program main
    use iso_fortran_env, only: dp => real64
    implicit none
    real(dp) :: sgn
    print *, sgn(10.0_dp)
end program main

function sgn(x) result(y)
    use iso_fortran_env, only: dp => real64
    implicit none
    real(dp),intent(in) :: x
    real(dp) :: y
    if (x>0.0_dp) then
        y = 1.0_dp
    else if (x<0.0_dp) then
        y = -1.0_dp
    else
        y = 0.0_dp
    end if
end function sgn
\end{verbatim}
但有个问题就是假如如果我们要算四精度, 只能另造个轮子.
\begin{verbatim}
program main
    use iso_fortran_env, only: dp => real64, qp => real128
    implicit none
    real(dp) :: sgn_real64
    real(qp) :: sgn_real128
    print *, sgn_real64(10.0_dp)
    print *, sgn_real128(10.0_qp)
end program main

function sgn_real64(x) result(y)
    use iso_fortran_env, only: dp => real64
    implicit none
    real(dp),intent(in) :: x
    real(dp) :: y
    if (x>0.0_dp) then
        y = 1.0_dp
    else if (x<0.0_dp) then
        y = -1.0_dp
    else
        y = 0.0_dp
    end if
end function sgn_real64

function sgn_real128(x) result(y)
    use iso_fortran_env, only: qp => real128
    implicit none
    real(qp),intent(in) :: x
    real(qp) :: y
    if (x>0.0_qp) then
        y = 1.0_qp
    else if (x<0.0_qp) then
        y = -1.0_qp
    else
        y = 0.0_qp
    end if
end function sgn_real128
\end{verbatim}
这就让我们有点小小的不开心, Fortran的语法也太严格了! 我们希望能造个\verb|sgn|, 又能算双精度又能算四精度. 用泛型接口, 我们就可以偷鸡摸狗地``做成''这件事, 把上面的\verb|sgn_real64|和\verb|sgn_real128| ``粘起来''.
\begin{verbatim}
program main
    use iso_fortran_env, only: dp => real64, qp => real128
    implicit none
    interface sgn
        function sgn_real64(x) result(y)
            use iso_fortran_env, only: dp => real64
            implicit none
            real(dp),intent(in) :: x
            real(dp) :: y
        end function sgn_real64
        function sgn_real128(x) result(y)
            use iso_fortran_env, only: qp => real128
            implicit none
            real(qp),intent(in) :: x
            real(qp) :: y
        end function sgn_real128
    end interface
    print *, sgn(10.0_dp)
    print *, sgn(10.0_qp)
end program main

function sgn_real64(x) result(y)
    use iso_fortran_env, only: dp => real64
    implicit none
    real(dp),intent(in) :: x
    real(dp) :: y
    if (x>0.0_dp) then
        y = 1.0_dp
    else if (x<0.0_dp) then
        y = -1.0_dp
    else
        y = 0.0_dp
    end if
end function sgn_real64

function sgn_real128(x) result(y)
    use iso_fortran_env, only: qp => real128
    implicit none
    real(qp),intent(in) :: x
    real(qp) :: y
    if (x>0.0_qp) then
        y = 1.0_qp
    else if (x<0.0_qp) then
        y = -1.0_qp
    else
        y = 0.0_qp
    end if
end function sgn_real128
\end{verbatim}
我们会发现泛型接口和特定接口很像, 只不过\verb|interface|后多了一串(示例中为\verb|sgn|). 这么写后, 电脑看到\verb|sgn(10.0_dp)|, 就会发现\verb|10.0_dp|是双精度的, 然后电脑就会在标\verb|sgn|的泛型接口里找, 发现\verb|sgn_real64|这个函数参量是双精度的, 匹配, \verb|sgn_real128|这个函数参量是四精度的, 不匹配, 于是乎电脑就会自动把\verb|sgn(10.0_dp)|里的\verb|sgn|当成\verb|sgn_real64|了, 然后电脑看到\verb|sgn(10.0_qp)|, 也是一样, 只不过最后是把\verb|sgn|当成\verb|sgn_real128|. 这样就仿佛有一个又能算双精度又能算四精度的\verb|sgn|了!

哑实结合时的类型--种别--维数匹配(type-kind-rank compatibility, TKR compatibility), 我们用泛型接口便能有所突破, 这当然是大好事. 但同志们还是可能不开心, 因为同志们会发现上面的\verb|sgn_real64|和\verb|sgn_real128|其实长得一模一样, 就是变量种别不同, 如果我们还要造单精度的轮子, 还要造整型的轮子, 岂不是要复制粘贴写一大堆一模一样的过程, 读起来还脑壳疼? 非常遗憾, 据我的了解, Fortran自己确实就只能这么玩儿了. 不过用预处理器(见第\ref{fortran_preprocessor}章)的话, 应该可以省事儿不少还易读, 但这已经是超Fortran的内容了.

\subsection{抽象接口}

特定接口只对应一个过程, 而抽象接口则对应所有特征(characteristic)相同的过程, 我们来看下面这个程序.
\begin{verbatim}
program main
    use iso_fortran_env, only: dp => real64
    implicit none
    interface
        function eye(n) result(mat)
            use iso_fortran_env, only: dp => real64
            implicit none
            integer,intent(in) :: n
            real(dp) :: mat(0:n-1,0:n-1)
        end function eye
        function minkowski(n) result(eta)
            use iso_fortran_env, only: dp => real64
            implicit none
            integer,intent(in) :: n
            real(dp) :: eta(0:n-1,0:n-1)
        end function minkowski
    end interface
    real(dp),dimension(0:3,0:3) :: mat_i, mat_m
    mat_i = eye(4)
    mat_m = minkowski(4)
end program main

function eye(n) result(mat)
    use iso_fortran_env, only: dp => real64
    implicit none
    integer,intent(in) :: n
    real(dp) :: mat(0:n-1,0:n-1)
    integer :: i
    mat = 0.0_dp

    do i = 0, n-1
        mat(i,i) = 1.0_dp
    end do
end function eye

function minkowski(n) result(eta)
    use iso_fortran_env, only: dp => real64
    implicit none
    integer,intent(in) :: n
    real(dp) :: eta(0:n-1,0:n-1)
    integer :: i
    eta = 0.0_dp
    eta(0,0) = -1.0_dp
    do i = 1, n-1
        eta(i,i) = 1.0_dp
    end do
end function minkowski
\end{verbatim}
这个程序, 接口块里写了两个过程的接口, 这当然没有问题. 但同志们会发现这两个过程其实接口长得``一样'', 都是函数, 都只有一个整型只读参量, 结果的类型, 种别和形状也一样, 只不过各种名称(过程名, 参量名, 结果名)不一样而已. 接口长得``一样''的过程, 我们称为特征相同的. 能不能趁机偷一小点懒? 能, 像下面这样.
\begin{verbatim}
program main
    use iso_fortran_env, only: dp => real64
    implicit none
    abstract interface
        function dim2mat(dim) result(mat)
            use iso_fortran_env, only: dp => real64
            implicit none
            integer,intent(in) :: dim
            real(dp) :: mat(0:dim-1,0:dim-1)
        end function dim2mat
    end interface
    procedure(dim2mat) :: eye, minkowski
    real(dp),dimension(0:3,0:3) :: mat_i, mat_m
    mat_i = eye(4)
    mat_m = minkowski(4)
end program main

function eye(n) result(mat)
    use iso_fortran_env, only: dp => real64
    implicit none
    integer,intent(in) :: n
    real(dp) :: mat(0:n-1,0:n-1)
    integer :: i
    mat = 0.0_dp
    do i = 0, n-1
        mat(i,i) = 1.0_dp
    end do
end function eye

function minkowski(n) result(eta)
    use iso_fortran_env, only: dp => real64
    implicit none
    integer,intent(in) :: n
    real(dp) :: eta(0:n-1,0:n-1)
    integer :: i
    eta = 0.0_dp
    eta(0,0) = -1.0_dp
    do i = 1, n-1
        eta(i,i) = 1.0_dp
    end do
end function minkowski
\end{verbatim}
和特定接口相比, 抽象接口的\verb|interface|前多了个\verb|abstract|. 接口块里过程名\verb|dim2mat|的接口和过程\verb|eye|, \verb|minkowski|的特定接口长得``一样'': 都是函数, 都只有一个整型只读参量等等. 至于抽象接口里函数名是\verb|dim2mat|, 参量名是\verb|dim|, 结果名是\verb|mat|, 无关紧要, 这些名称都是可以随便取的\footnote{如果碰上了什么规矩, 编译器会告诉我们.}, 只要在接口块里一致即可. 然后我们只需要再写\verb|procedure(dim2mat) :: ...|就相当于给\verb|...|加特定接口了, 但前提是\verb|...|必须是和\verb|dim2mat|特征相同的过程(如\verb|eye|和\verb|minkowski|).

在第\pageref{dummy_procedure_program}页的轮子里, 我在声明哑过程的时候用了抽象接口, 这个地方, 根据我自己的实验, 非用抽象接口不可. 按我的理解, 这是因为哑过程根本不是一个``真''过程: 既不是我们用代码定义的过程, 也不是编译器定义好的固有过程. 只有``真''过程能有特定接口, 因为电脑碰到特定接口的时候, 会去查特定接口对应的过程的定义, 而哑过程只有声明没有定义, 所以声明哑过程的时候就只能用抽象接口了.
